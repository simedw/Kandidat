\documentclass[Rapport]{subfiles}
\begin{document}

\section{Relaterade Arbeten}
Vårt arbete spänner över två forskningsområden, partiell evaluering
och optimering under körningstid. I litteraturen är dessa områden
oftas separerade, något som vi håller'


\subsection{Statiska optimeringar}

Core\cite{mitchell2007supercompiler}

Partiell evaluering


\subsection{Optimeringar under körningstid}

JIT (just in time) compilation är en metod för att dynamiskt optimera
de delar av ett program som används mest under körningstid. Val av
dessa punkter sker med hjälp av statistik och heuristik gentemot vår
lösning där programmeraren själv har kontroll över angreppspunkterna. 

Att annotera optimeringspunkten har gjorts tidigare av Bolz C., då
för en Prologvariant.\cite{bolz-automatic} Vårt arbete liknar hans
på många punker, men skiljer sig i bl. a. hur vi hanterar förgreningar
(i vårt fall case-satser i hans if-satser). Han lägger till en callback
medan vi försöker fortsätta optimeringen.


\subsection{Funktionella maskiner}

Stg....\cite{marlow2006making}

		% \newpage

%\bibliographystyle{apalike}
\bibliography{references}

\end{document}

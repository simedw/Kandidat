\documentclass[Rapport]{subfiles}
\begin{document}

\section{Sockerspråk}

%
% Bnf, exempel
%
% socker
%
% lambdalyftare
%
% definitioner av tex toppnivå, substitution
%
% letrec
%
% vad är en PAP
%

%skiss:
%
%haskell 
%    \
%     \
%      \
%       \
%        \        sockerspråk
%         \          /
%          \        /
%           \      /
%            \ STG

\subsection{Introduktion}

För att undersöka optimeringsmöjligheterna behövs ett språk att arbeta med. 
Detta språk behöver inte vara lika rikt som Haskell, men tillräckligt bekvämt
att arbeta med för att skriva några större program. Språket som används i den
här rapporten kommer att refereras till som sockerspråket, för att det 
innehåller syntatiskt socker. Socker är ett bekvämt och ofta mer kompakt sätt 
att skriva diverse saker på. Ett klassiskt exempel är syntaxen
för listor i Haskell, \miniCode{[5,0,4]} som är socker för \miniCode{5 : 0 : 4 : []}.
    Det går att översätta Haskell till sockerspråket, och naturligtvis också
i motsatt riktning. Sockerspråket är också designat så, att det är lätt att
översätta och avsockras till core-språket som introduceras i nästa kapitel. 

\subsection{BNF}
\NOTE{BNFen ser mycket dålig ut. Detta ska naturligtvis lösas.}

\begin{equation*}
\begin{aligned}
program ::=&\; \overline{function} \\
function ::=&\; v \; \overline{v} \mathtt{=} expr \mathtt{;} \\
expr ::=&\;  v \\
       |&\; v \; \overline{expr}  \\
       |&\; \mathtt{let \{} \overline{function} \mathtt{\} in }\; expr \\
       |&\; \mathtt{letrec \{} \overline{function} \mathtt{\} in} \; expr \\
       |&\; \mathtt{case}\,expr\,\mathtt{of \{} \overline{branch} \mathtt{\}} \\
       |&\; \mathtt{\lambda } v \mathtt{.} expr \\
       |&\; expr \oplus expr \\
       |&\; K \; \overline{expr} \\
       |&\; primitive \\
branch ::=&\; K \; \overline{v} \; \mathtt{\rightarrow} \; expr \mathtt{;} \\
         |&\; v \; \mathtt{\rightarrow} \; expr \mathtt{;}
\end{aligned}
\end{equation*}

Programmet är en lista av funktioner. Dessa funktioner är de som sägs vara på
\emph{toppnivå}. Funktioner som är skapade med en lambdaabstraktion eller i en
letsats är inte på toppnivå. 
  Funktionerna har en identifierare, dess namn, med noll eller fler variabler som argument. Efter likhetstecknet
tillkommer deras funktionskropp, som är ett uttryck \miniCode{expr}, och funktionen avslutas med
ett semikolon.
    Uttrycken kan antingen vara en variabel, ett funktionsanrop som är ett
funktionsnamn applicerat på en eller fler uttryck. Det kan också vara let eller
letrec-bundna funktioner (eller variabler i det fallet då de är funktioner som
inte tar något argument), eller en casesats som tar ett uttryck och flera grenar.
Som uttryck tillåts också primitiva operationer som skrivs infix, tex addition 
och multiplikation. Att returnera en konstruktor eller en primitiv som 
exempelvis ett heltal eller decimaltal är också ett uttryck.
    Grenarna kan antigen vara att mönstermatcha mot en konstruktor och binda
dess konstruktionsvariabler, eller att matcha vilken konstruktor som helst.


\subsection{Jämförelse med Haskell}

    En feature som saknas i sockerspråket som ofta används flitigt i Haskell är 
möjligheten att mönstermatcha närhelst en variabel binds, tex i vänsterledet
i en funktionsdefinition. Det är hursomhelst så att alla mönstermatchningar,
även med vakter, går att översätta till enkla casesatser. Vårt sockerspråk
har endast sådana. De är enkla i det avseendet att man inte kan mönstermatcha i 
mer än ett djup, och att det inte finns några vakter.

Här är insertmetoden skriven i Haskell, den sätter in ett element i en redan sorterad
lista utan att förstöra sorteringen.

\begin{codeEx}
  insert :: Ord a => a -> [a] -> [a]
  insert v (x:xs) 
      | v < x     = v:x:xs
      | otherwise = x:insert v xs 
  insert v [] = [v]
\end{codeEx}                  

I sockerspråket skrivs samma funktion som

\begin{codeEx}
  insert v list = case list of
      Cons x xs -> case v < x of
          True  -> v:x:xs
          False -> x:insert v xs
      Nil -> v:Nil
\end{codeEx}

    I Haskell definieras användarnas datatyper explicit, som i definitionen
av Maybe:
\begin{codeEx}
data Maybe a = Just a | Nothing
\end{codeEx}
I sockerspråket finns ingen sådan konstruktion, utan det är bara att använda
konstruktorerna Just och Nothing när så behövs.


Sockerspråket gör åtskillnad på rekursiva och vanliga let-bindningar. 
\begin{codeEx}
repeat x = letrec xs = x : xs in xs  
\end{codeEx}
Här används xs, som binds av letrec-satsen, även i uttrycket efteråt. Detta
finns inte i Haskell, då alla let-bindningar räknas som rekursiva. I
sockerspråket får användaren också ha i åtanke att let-bundna variabler binds
sekvensiellt, dvs att de måste skrivas i ordning:
\begin{codeEx}
let { t1 = f x y
    ; t2 = g y t1
    }
in  h t1 t2
\end{codeEx}
Här går det inte, till skillnad från i Haskell, att byta plats på t1 och t2.

\subsection{Standardbibliotek och testsvit}

För att slippa skriva om alla enkla funktioner hela tiden gjorde vi ett litet standardbibliotek
som helt enkelt är en fil med en rad funktioner skriven i sockerspråket som läggs
till på slutet av den aktuella filen som ska köras.

Under tiden som vi jobbade med projektet skrev vi ofta små enkla
program för att testa ny funktionalitet. Vi samlade på oss en mängd sådana program av
olika karaktär och komplexitet och gjorde så att de kunde testas automatiskt, 
vilket var ett bra redskap för oss 
när vi gjorde ändringar i tolken eftersom vi då kunde se om något som tidigare
fungerat slutade fungera med en viss ändring. 

En testsvit är också en bra morot när man gör stora ändringar som kanske gör så att
alla testprogram slutar fungera, för när man så sakteliga får saker att fungera
igen kommer det att reflekteras i antalet fungerande testprogram i testsviten. 
Man ser konkreta resultat av sitt arbete.



\subsection{Definitioner}

\subsubsection{Fria variabler och substitution}
Låt oss säga att vi har en funktion f som använder sina två 
argument, x och y. Om vi betraktar funktionskroppsuttrycket för sig är x och y 
\emph{fria variabler}. Det skulle inte gå att evaluera funktionskroppen utan
att känna till vad de är. Känner man till x och y:s värden, låt oss säga
$x \mapsto True$ och $y \mapsto 9$, så kan man \emph{substituera} in dem i uttrycket. Då byts varje
x ut mot True och varje y mot 9. Om uttrycket är $e$ skrivs en sådan substituering som $e[True/x, 9/y]$. 
    Ett sätt att konkret realisera en substitution är att traversera trädet
för uttrycket och jämföra varje variabel och ersätta om det är en av de
relevanta. Detta är den effekten man alltid vill eftersträva rent abstrakt.
Det går också att tänka sig andra sätt att implementera substitution, genom
att tex göra ett uppslag i en tabell varje gång en fri variabel funnes när
koden evalueras. 

\subsubsection{Partiell evaluering}
Detta arbete handlar till stor del om partiell evaluering. Men vad innebär det?
I funktionella språk är funktioner första-ordningens medlemmar och kan tex
skickas med som argument till andra funktioner, eller så kan de returneras från
andra funktioner. Tag tex zipWith, här skriven i Haskell:
\begin{codeEx}
zipWith :: (a -> b -> c) -> [a] -> [b] -> [c]
zipWith f (x:xs) (y:ys) = f x y : zipWith f xs ys
zipWith _ _      _      = []
\end{codeEx}
Vad kan f vara här? Det kan tex vara en unär funktion som returnerar en ny
funktion. Det kan vara et funktion som tar tre argument och redan har ett
applicerat:
\begin{codeEx}
let f = \x y z -> x * y - z
in  zipWith (f 3) list1 list2
\end{codeEx}
Funktioner som har några, men inte alla argument applicerade kallas 
partiellt applicerade. Att börja evaluera en sådan funktion, partialevaluera,
är inget som vanligtvis görs i Haskellimplementationer,
och vad som undersöks i detta projekt. 

\end{document}

\documentclass[Rapport]{subfiles}
\begin{document}

\chapter{Sockerspråk}
%\section{Sockerspråk}
\label{sec:Socker}

%
% Bnf, exempel
%
% socker
%
% lambdalyftare
%
% definitioner av tex toppnivå, substitution
%
% letrec
%
% vad är en PAP
%

%skiss:
%
%haskell 
%    \
%     \
%      \
%       \
%        \        sockerspråk
%         \          /
%          \        /
%           \      /
%            \ STG

% överflödig övergångstext
% Som nämnt i inledningen har kommer vi arbeta med en maskin och tillhörande
% språk som heter STG, och för att göra det vänligare att arbeta med har 
% sockerspråket utvecklats, med ett lager socker ovanpå STG-språket. Detta 


%\subsection{Introduktion}

%Figur \ref{figure:overviewSugar} visar var vi befinner oss just nu:

\overviewSugar

Som figur \ref{figure:overviewSugar} visar är vi alltså i början av tolkens pipeline, där kod skriven i sockerspråket parsas till ett abstrakt syntaxträd.
% vid den kod som är skriven i sockerspråket
%och parsats till ett abstrakt syntaxträd.

För att undersöka optimeringsmöjligheterna behövs ett språk att arbeta med. 
Detta språk behöver inte vara lika rikt som Haskell, men bekvämt nog
att arbeta med för att skriva några större program. Språket som används i den
här rapporten kommer att refereras till som sockerspråket, för att det 
innehåller syntaktiskt socker. Socker är ett behändigt och ofta mer kompakt sätt 
att skriva kod på. Ett klassiskt exempel är syntaxen
för listor i Haskell: \miniCode{[5,0,4]} är socker för \miniCode{5 : 0 : 4 : []}.

Sockerspråket är så likt Haskell att det går enkelt att översätta mellan språken.
Det är också designat så att det är lätt att översätta och avsockra det till 
STG-språket som introduceras i nästa kapitel. I det här kapitlet
förklaras istället hur sockerspråket ser ut och hur det används, och hur
det skiljer sig från Haskell.

Det finns ingen typcheckare till sockerspråket, men då den här optimeringen
är tänkt att fungera i Haskell, som är statiskt typat och typcheckat, antas
alla program som körs vara typkorrekta. Vi kommer inte att resonera om några program
som använder dynamisk typning eller typning som inte skulle kunna gå
att skriva i Haskell.

\section{Grammatik}

Så här ser BNF\footnote{Backus–Naur Form}-grammatiken ut för sockerspråket:

\begin{figure}[H]
\begin{equation*}
\begin{aligned}
Variabler \quad & v,f \\
Konstruktor \quad & K \\
Program \quad& p &::=&\; \overline{def} 
        & \text{Program är flera definitioner}\\ 
Definitioner  \quad & def & ::=&\; f \; \overline{v} \, \mathtt{=} \, e \mathtt{;} 
        & \text {Bind funktionsuttryck}\\
Uttryck \quad & e &::=&\;  v \\
       &&|&\; f \; \overline{e}  
            & \text{Funktionsanrop} \\ 
       &&|&\; \mathtt{let \, \{} \, \overline{def} \, \mathtt{\} \, in }\; e 
            & \text{Lokala definitioner}\\
       &&|&\; \mathtt{letrec \, \{} \, \overline{def} \, \mathtt{\} \, in } \; e 
            & \text{Lokala rekursiva definitioner} \\
       &&|&\; \mathtt{case} \, e \, \mathtt{of \, \{} \, \overline{brs} \, \mathtt{\}} 
            & \text{Mönstermatchning av uttryck} \\
       &&|&\; \mathtt{\lambda } v \mathtt{.} e
            & \text{Anonym funktion} \\
       &&|&\; e \oplus e 
            & \text{Primitiv operation}\\ 
       &&|&\; K \; \overline{e} 
            & \text{Konstruktorapplikation} \\
       &&|&\; primitiv 
            & \text{Primitiv datatyp} \\
       &&|&\; \mathtt{optimise} \; e 
            & \text{Optimera uttrycket} \\
       &&|&\; \mathtt{optimise } \; e \; \mathtt{ with \{} \, \text{direktiv} \, \} 
            & \text{Optimera uttrycket} \\
Alternativ \quad & brs &::=&\; K \; \overline{v} \; \mathtt{\rightarrow} \; e \mathtt{;} 
            & \text{Matchning mot konstruktor} \\
         &&|&\; v \; \mathtt{\rightarrow} \; e \mathtt{;}
            & \text{Matchning med en variabel}
\end{aligned}
\end{equation*}
\caption{Sockerspråkets syntax}
\end{figure}

Ett program är en lista av definitioner. Att det är en lista visas med ett
streck över namnet, som i $\overline{def}$.
Dessa funktionsdefinitioner är de som sägs vara på \emph{toppnivå}. Funktioner 
skapade med lambdaabstraktion eller \kw{let}-sats är ej på toppnivå. 
\begin{itemize}
\item En funktion har en identifierare, ett namn och noll eller fler variabler som argument. Efter likhetstecknet
kommer funktionskroppen, som är ett uttryck \miniCode{e}, och den avslutas med
ett semikolon.

\item  Ett uttryck kan vara en variabel eller ett funktionsanrop, som är ett
funktionsnamn, applicerat på ett eller flera uttryck. Det kan också vara en \kw{let}-sats eller
\kw{letrec}-bundna definitioner (eller variabler om de inte tar något argument).
Man kan ha \kw{case}-satser som tar ett uttryck och flera grenar, vilket kommer
att evaluera uttrycket och sedan välja den gren som matchar.

Som uttryck tillåts också primitiva operationer som skrivs infix, exempelvis addition 
och multiplikation. Konstruktorer och primitiver (heltal, decimaltal och strängliteraler) är också uttryck.

%Det sista uttrycken är de som har med optimeringen att göra, de kan ha olika direktiv
%som bestämmer hur optimeringen skall gå till. Mer om vilka direktiv som finns går
%att läsa i sektion \ref{sec:Optimise:With}.

\item  Det sista uttrycket, \kw{optimise ...}, tar ett uttryck som ska optimeras. Det finns även direktiv för hur optimeringen ska gå till, som går att läsa om i sektion \ref{sec:Optimise:With}.


\item En gren består antingen av en mönstermatchning med en konstruktor där
      dess konstruktionsvariabler binds, eller en variabel som kan matcha mot vad som helst.
\end{itemize}

\section{Jämförelse med Haskell}

    En egenskap som saknas i sockerspråket men som ofta används flitigt i Haskell är 
möjligheten att mönstermatcha närhelst en variabel binds, till exempel i vänsterledet
av en funktionsdefinition. Det är hursomhelst så att alla mönstermatchningar,
även med vakter, går att översätta till enkla \kw{case}-satser. Vårt sockerspråk
har endast dessa. De är enkla i det avseendet att man inte kan mönstermatcha djupare
än en nivå, och att det inte finns några vakter.

Här är insertmetoden skriven i Haskell, vilken sätter in ett element i en redan sorterad
lista utan att förstöra sorteringen:

\begin{codeEx}
  insert :: Ord a => a -> [a] -> [a]
  insert v (x:xs) 
      | v <= x    = v : x : xs
      | otherwise = x : insert v xs 
  insert v [] = [v]
\end{codeEx}                  

I sockerspråket skrivs samma funktion istället på det här viset:

\begin{codeEx}
  insert v list = case list of
      { Cons x xs -> case v <= x of
          { True  -> v : x : xs
          ; False -> x : insert v xs
          }
      ; Nil -> v : Nil
      }
\end{codeEx}

I Haskell kan programmeraren explicit definiera egna datatyper, som exempelvis \miniCode{Maybe}:
    
\begin{codeEx}
data Maybe a = Just a | Nothing
\end{codeEx}

I sockerspråket finns ingen sådan konstruktion och programmeraren kan därmed inte
definiera datatypen explicit. Istället är det bara att använda
konstruktorerna \miniCode{Just} och \miniCode{Nothing} när de behövs.


Sockerspråket skiljer på rekursiva och vanliga \kw{let}-bindningar. 

\begin{codeEx}
repeat x = letrec xs = x : xs in xs;
\end{codeEx}

Här används \miniCode{xs}, som binds av \kw{letrec}-satsen, även i uttrycket i högerledet.
Nyckelordet \kw{letrec} finns inte i Haskell, där istället alla \kw{let}-bindningar räknas som rekursiva. I
sockerspråket måste användaren också tänka på att \kw{let}-bundna variabler binds
sekvensiellt, och därför måste skrivas i ordning:

\begin{codeEx}
let { t1 = f x y
    ; t2 = g y t1
    }
in  h t1 t2
\end{codeEx}

%Här går det inte, till skillnad från i Haskell, att byta plats på \kw{t1} och \kw{t2}.
Vilket innebär, till skillnad från hur det fungerar i Haskell, att det inte går att byta plats på \kw{t1} och \kw{t2}.

Sammanfattningsvis är sockerspråket nästan en delmängd av Haskell. Ett program
skrivet i sockerspråket kan köras som Haskell-kod med följande förändringar:
\begin{itemize}
    \item \kw{data}-deklarationer läggs till
    \item nyckelordet \kw{letrec} ändras till \kw{let}
    \item lambdauttryck byts från formen \ic{(\textbackslash x . x)} till \ic{(\textbackslash x -> x)}
          %AGDA-STYLE MIXFIX
    \item \kw{if\_then\_else\_}-satser används 
          istället för den treställiga \ic{if}-funktionen 
\end{itemize}
Det finns också vissa skillnader i de olika standardbiblioteken, I Haskell 
kallas detta \emph{Prelude}.
Den främsta skillnaden är att alla funktioner som opererar på listor använder 
konstruktörerna \ic{Cons} och \ic{Nil} i sockerspråket
där Haskell använder \ic{(:)} och \ic{[]}.

    En viktig semantisk skillnad mellan sockerspråket och Haskell är att
\kw{case}-satser i sockerspråket alltid evaluerar uttrycket till WHNF (se
kapitel \ref{sec:whnf}) medan det i Haskell inte är någon semantisk skillnad mellan att
\kw{case}-granska på ett uttryck eller att mönstermatcha på det i högerledet
i en funktion- eller let-definition. I Haskell forcerar ingen av dessa någon
evaluering. 

% Dan: Jag känner att jag har tappat min tilltro till casesastser nu.
%      Jag vet inte om jag någonsin kommer våga förlita mig på dessa igen.

\end{document}
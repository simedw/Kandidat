\documentclass[Rapport]{subfiles}
\begin{document}

\section{Inledning | Bakgrund och motivering}

Vanligtvis optimeras kod endast vid kompileringsfasen. I följande Haskell-exempel 
skulle en sådan optimering inte ge ett optimalt resultat.

\begin{codeEx}
power :: Integer -> Integer -> Integer
power 0 x = 1
power n x = x * power (n - 1) x

powers :: Integer -> [Integer] -> [Integer]
powers n xs = map (power n) xs

main = do
    n <- getIntegerFromUserInput
    print (powers n [1..100])
\end{codeEx}

Funktionen $powers$ tar in ett tal $n$ och en lista $xs$, och kör funktionen $power\:n$, 
som upphöjer ett tal till $n$, på varje element i listan. Om indatan $n$ till 
$powers$ är känd statiskt (när programmet kompileras) skulle en kompilator kunna 
göra en specialiserad version av $power$ för just det $n$-värdet. Till exempel, när $n=4$:

\begin{codeEx}
power_4 :: Integer -> Integer
power_4 x = x * x * x * x
\end{codeEx}

Den specialiserade versionen av $power$ är snabbare än originalet, då den inte är 
rekursiv och inte behöver titta på värdet som $n$ har.

I $main$-funktionen ovan är dock $n$ inte känt när programmet kompileras, utan läses 
istället in från användaren under körningstid. Detta gör att en kompilator inte 
har tillräckligt mycket information för att kunna göra en specialiserad version av $power$.

Vi har i det här projektet tagit fram en möjlig lösning på detta problem, där programmeraren kan 
annotera var optimeringar ska ske under körningstid. 


\end{document}

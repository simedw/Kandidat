\documentclass[../Optimise]{subfiles}
\begin{document}

\numberwithin{equation}{section}

\subsection{Call by Value semantik}

\subsubsection{Funktionsapplikationer}

\begin{align}
\label{CBV:Fun1} \OMEGA sH{f\,\bar{k}}  \Rightarrow &\, \mc{f\,\bar{k}}sH\\
 & \overline{k} \text{ all known} \nonumber \\
\label{CBV:Fun2} \OMEGA sH{f\,\bar{x}\, k\,\bar{y}}  \Rightarrow &\, \mc{f\,\bar{x}\, k\,\bar{y}}{\oInline s}H\\
 & \overline{x},\overline{y} \text{ unknown, } k \text{ is known} \nonumber \\
\label{CBV:Fun3} \IRR{\cOFUN{\bar{x}}{\alpha}s}He  \Rightarrow &\, \mc{\alpha}s{\heap{\alpha}{\FUN{\bar{x}}e}}
\end{align}

I regel \eqref{CBV:Fun1} är alla argument till funktionen kända, då låter vi STG applicera funktionen precis som under vanlig körning. I \eqref{CBV:Fun2} är inte
alla argument kända, då inline:as endast de kända argumenten till funktionen. 
När vi har optimerat en funktionskropp så lång det går bygger vi upp funktionen igen \eqref{CBV:Fun3}.


\subsubsection{let-uttryck}

\begin{align}
\label{CBV:Let1} \OMEGA sH{\eLet x{\THUNK e}{e'}}  \Rightarrow &\, \OMEGA{\oLet x{e'}s}He\\
\label{CBV:Let2} \PSI{\oLet xes}Hv  \Rightarrow &\, \OMEGA sH{e[v/x]} \\
\label{CBV:Let3} \IRR{\oLet xes}H{e'}  \Rightarrow &\, \OMEGA{\oLetObj x{\THUNK{e'}}s}He
\end{align}

Regel \eqref{CBV:Let1} innebär att för varje thunk vi möter kommer vi försöka optimera
dess innehåll. Detta är huvudanledning till många av våra problem som diskuteras i \ref{CBV:Problem}.
I \eqref{CBV:Let2} har optimering lyckats evaluer fram ett värde ifrån thunken, därmed är thunken inte längre nödvändig och vi substituerar in dess värde i uttrycket. Om vi däremot inte lyckads få fram ett värde måste vi bygga upp let:en igen \eqref{CBV:Let3}, men iställt för att bygga upp den med en gång lägger vi ut en ny continuation och försöker optimera uttrycket efter \kw{let .. in}.


\begin{align}
\label{CBV:OLet1} \OMEGA sH{\eLet x{\CON{C}{\overline{as}}}{e}}  \Rightarrow &\, \OMEGA{\oLetObj x{OBJ}s}{\heap{v}{\CON{C}{\overline{as}}}}e\\
 & \overline{as} \text{ all known, } v \text{ fresh} \nonumber \\
\label{CBV:OLet2}\OMEGA sH{\eLet x{OBJ}{e}}  \Rightarrow &\, \OMEGA{\oLetObj x{OBJ}s}He\\
\PSI{\oLetObj x{OBJ}s}Hv  \Rightarrow &\, \IRR sH{\eLet x{OBJ}v}\\
\IRR{\oLetObj x{OBJ}s}He  \Rightarrow &\, \IRR sH{\eLet x{OBJ}e}
\end{align}

Om objektet inte var en thunk kan två olika saker ske, \eqref{CBV:OLet1} är ifall 
vi har en känd \kw{CON}. Då allokerar vi den på heapen på samma sätt som maskinen
och fortsätter att optimera det inre uttrycket. I regel \eqref{CBV:OLet2} så 
fortsätter vi bara optimera med det inre uttrycket.

$\Psi$ samt $\Phi$ bygger bara upp \kw{let}-uttrycket igen och går vidare med $\Phi$
då det betyder att vi har optimerat klart. Ett problem här är att om man har en \kw{let}
i en \kw{case}-granskare så kommer vi inte välja rätt gren, även om vi har ett värde.

\begin{codeEx}
foo x = case (let y = CON (X 2) in y) of
    { X z -> ..
    };
\end{codeEx} 

I detta fall vet vi att värdet på granskaren är ett X och z är 2, men vår optimering
kommer inte att se detta. Det här är ytterligare en brist som den vår senare semantik
kommer att lösa.

\subsubsection{Case}
\begin{align}
\OMEGA sH{\eCase e{brs}} \Rightarrow &\, \OMEGA{\oCase{brs}s}He\\
\label{CBV:Case2} \PSI{\oCase{brs}s}Hv \Rightarrow &\, \OMEGA sH{\text{instantiate correct brs with v}} \\
\label{CBV:Case3} \IRR{\oCase{brs}s}He \Rightarrow &\, \IRR sH{\eCase e brs}
\end{align}

Instantiate correct branch i regel \eqref{CBV:Case2} väljer helt enkelt vilken branch som matchar
värdet som maskinen arbetat fram. Detta är troligtvis en av de viktigaste
stegen i vår optimering då kan skala bort alla onödiga brancher, minska koden 
och oftast fortsätta optimera en bit in det nya uttrycket.

Istället för att skapa en $\eCase e brs$ i \eqref{CBV:Case3} har vi även testat att gå in
och optimera varje branch, detta bryter dock semantiken, vi kan fastna
i oändliga loopar i brancher vi aldrig hade hamnat i. 

\begin{codeEx}
case x of
    { True  -> 5 : Nil
    ; False -> repeat 1
    }
\end{codeEx}

\subsubsection{Atoms}
\begin{align}
\OMEGA s{\heap{v}{\THUNK{e}}}v \Rightarrow &\, \OMEGA sHe
\end{align}

Vi har bara en regel för atoms och det är ifall vi hittar en variabel som på heapen
är en thunk, så börjar vi optimera vad thunken har bundit.

\subsubsection{STG interface}
När vi interagerar med STG maskinen måste vi har tydliga regler
för när maskien ska sluta evaluera och börja optimera. Detta sker
antingen då den stöter på en $\OPT t$ för första gången, eller när den
inte längre kan evaluera ett uttryck och har en Optimise continuation
på stacken.

\begin{align}
\mc as{\heap a{\OPT t}} & \Rightarrow\\
\mc t{\cOPT{a\,}s}{\heap a{\BH}}\\
\mc x{\cOPT as}{\heapp x{\PAP f{a_{1}\cdots a_{n}}}f{\FUN{x_{1}\cdots x_{m}}e}} & \Rightarrow\\
\OMEGA{\cOFUN{x_{n+1}\cdots x_{m}}as}H{e[a_{1}/x_{1}\cdots a_{n}/x_{n}]}\\
\mc v{O.\star\,:\, s}{\nheap v{\THUNK e}} & \Rightarrow\\
\PSI{O.\star\,:\, s}Hv
\end{align}

\NOTE {
\begin{align*}
\OMEGA sH{\eCase e{brs}} & \Rightarrow & \OMEGA{\oCase{brs}s}He\\
\OMEGA sH{f\,\bar{k}} & \Rightarrow & \mc{f\,\bar{k}}sH\\
\OMEGA sH{f\,\bar{x}\, k\,\bar{y}} & \Rightarrow & \mc{f\,\bar{x}\, k\,\bar{y}}{\oInline s}H\\
\OMEGA sH{\eLet x{\THUNK e}{e'}} & \Rightarrow & \OMEGA{\oLet x{e'}s}He\\
\OMEGA s{\nheap x{\,}}{\bullet} & \Rightarrow & \IRR sH{\bullet}
\end{align*}

\begin{align*}
\PSI{\oLet xes}Hv & \Rightarrow & \OMEGA sH{e[v/x]}
\end{align*}

\begin{align*}
\PSI{\oLetObj x{OBJ}s}Hv & \Rightarrow & \IRR sH{\eLet x{OBJ}v}\\
\PSI{\oCase{brs}s}Hv & \Rightarrow & \OMEGA sH{\text{instantiate correct brs with v}}\end{align*}

\begin{align*}
\IRR{\oLet xes}H{e'} & \Rightarrow & \OMEGA{\oLetObj x{\THUNK{e'}}s}He\\
\IRR{\oCase{brs}s}He & \Rightarrow & \IRR sH{\eCase e brs}
\end{align*}

\begin{align*}
\IRR{\oLetObj x{OBJ}s}He & \Rightarrow & \IRR sH{\eLet x{OBJ}e}\\
\IRR{\cOFUN{\bar{x}}{\alpha}s}He & \Rightarrow & \mc{\alpha}s{\heap{\alpha}{\FUN{\bar{x}}e}}
\end{align*}

När vi interagerar med STG maskinen måste vi har tydliga regler
för när maskien ska sluta evaluera och börja optimera. Detta sker
antingen då den stöter på en $\OPT t$ för första gången, eller när den
inte längre kan evaluera ett uttryck och har en Optimise continuation
på stacken.

\begin{align*}
\mc as{\heap a{\OPT t}} & \Rightarrow\\
\mc t{\cOPT{a\,}s}{\heap a{\BH}}\\
\mc x{\cOPT as}{\heapp x{\PAP f{a_{1}\cdots a_{n}}}f{\FUN{x_{1}\cdots x_{m}}e}} & \Rightarrow\\
\OMEGA{\cOFUN{x_{n+1}\cdots x_{m}}as}H{e[a_{1}/x_{1}\cdots a_{n}/x_{n}]}\\
\mc v{O.\star\,:\, s}{\nheap v{\THUNK e}} & \Rightarrow\\
\PSI{O.\star\,:\, s}Hv
\end{align*}

}

%\begin{multline}
%%\mc as{\heap a{\OPT t}} \Rightarrow \mc t{\cOPT{a\,}s}{\heap a{\BH}} \\
%\mc x{\cOPT as}{\heapp x{\PAP f{a_{1}\cdots a_{n}}}f{\FUN{x_{1}\cdots x_{m}}e}} \\ \Rightarrow  \OMEGA{\cOFUN{x_{n+1}\cdots x_{m}}as}H{e[a_{1}/x_{1}\cdots a_{n}/x_{n}]}\\
%%\mc v{O.\star\,:\, s}{\nheap v{\THUNK e}} & \Rightarrow & \PSI{O.\star\,:\, s}Hv
%\end{multline}

\subsubsection{Problem med CBV semantiken}
\label{CBV:Problem}

Oturligt nog bröt ovanstående optimise semantik mot STGs egna semantik. 
Som bekant forceras evaluering av thunkar endast i case-granskaren. 
Om funktionen som optimeras hade forcerat en thunk under vanlig körning så är
 opimise också tillåten att göra detta. Det finns ibland anledning att focera
 även vid andra tillfällen, men vårt naiva tillvägagångsätt höll inte måttet.

\begin{codeEx}
take n list = if (n == 0) Nil (head list : take (n-1) (tail list))
\end{codeEx}
Här vill man först evaluera \ic{n == 0} innan någon av argumenten till \ic{if}.
If är defienierad med en \kw{case} sats på det första argumentet. Även om
if inlinas så ligger inte casen-först i funktionen. För att bättre förstå varför kan vi observera hur take koden ser ut osockrad. 

\begin{codeEx}
take n list = let 
    { t1 = THUNK (n == 0)
    ; t2 = THUNK (head list)
    ; t3 = THUNK (n - 1)
    ; t4 = THUNK (tail list)
    ; t5 = THUNK (take t3 t4)
    } in  if t1 Nil t5;
\end{codeEx}

Vi evaluerar varje ny thunk efter vi passerat den, när vi kommer till 
\miniCode{t5} görs ett rekursivt anrop till take och processen börjar om.
\miniCode{if t1 Nil t5} får aldrig en chans att avbryta loopen och optimeringen terminerar aldrig.

Definierades take istället på följande sättet terminerar optimise:

\begin{codeEx}
take n list = case n == 0 of
    { True -> Nil
    ; False -> let 
        { t2 = THUNK (head list)
        ; t3 = THUNK (n - 1)
        ; t4 = THUNK (tail list)
        } in  take t3 t4
    };
\end{codeEx}

%\NOTE{oeftersträvansvärt var ordet}

Och det är oeftersträvansvärt att tvinga användarna att skriva all funktioner
de vill optimera med detta i åtanke. Användaren behöver inte bara hålla reda på 
hur sina funktioner är skrivan, men också hur alla funktioner han använder är skrivna. 
Detta arbetar emot principen om abstraktion vilket är oturligt. Detta leder oss in på vår strävan
efter en lat optimeringssemantik.


\end{document}

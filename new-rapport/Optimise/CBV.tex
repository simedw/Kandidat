\documentclass[../Optimise]{subfiles}
\begin{document}

\subsection{Call-by-Value-semantik}

\NOTE{
Dessa regler är listade lite fult samt det förklaras inte riktigt varför
de ser ut som de gör eller varför de fungerar 
}

Det finns tre delar i denna form av optimise semantik. De är som följer:

\textbf{Omega} $\Omega$, är ingångspunkten till optimeringsfunktionen, anropas från maskinen
när något har blivit annoterat för optimering. Omega bryter ner trädet
och delegerar vidare vem och vad som ska ske, samt lägger ut continuations
för att senare bygga upp ett förhoppningsvis nytt träd.

\begin{align*}
\OMEGA sH{\eCase e{brs}} & \Rightarrow & \OMEGA{\oCase{brs}s}He\\
\OMEGA sH{f\,\bar{k}} & \Rightarrow & \mc{f\,\bar{k}}sH\\
\OMEGA sH{f\,\bar{x}\, k\,\bar{y}} & \Rightarrow & \mc{f\,\bar{x}\, k\,\bar{y}}{\oInline s}H\\
\OMEGA sH{\eLet x{\THUNK e}{e'}} & \Rightarrow & \OMEGA{\oLet x{e'}s}He\\
\OMEGA s{\nheap x{\,}}{\bullet} & \Rightarrow & \IRR sH{\bullet}
\end{align*}


\textbf{Psi} $\Psi$, anropas från när optimise har lagt ut något åt maskinen och
maskinen inte kan komma längre. Psi kommer använda värdet från maskinen
och sedan delegera vidare optimeringen.

\begin{align*}
\PSI{\oLet xes}Hv & \Rightarrow & \OMEGA sH{e[v/x]}
\end{align*}

\NOTE{Viktigt att nämna att substitute är långsamt?
eller skall detta vara mer generellt och inte bry sig om hur e[a/x] fungerar}

Här byter vi ut x mot värdet v som maskinen har arbetat fram let-uttrycket.
Här använder vi en substitute funktion som går igenom hela e uttrycksträd
vilket är väldigt kostsamt. Något som vi bytte ut i senare versioner.

\begin{align*}
\PSI{\oLetObj x{OBJ}s}Hv & \Rightarrow & \IRR sH{\eLet x{OBJ}v}\\
\PSI{\oCase{brs}s}Hv & \Rightarrow & \OMEGA sH{\text{instantiate correct brs with v}}\end{align*}


Instantiate correct branch väljer helt enkelt vilken branch som matchar
värdet som maskinen arbetat fram. Detta är troligtvis en av de viktigaste
stegen i vår optimering då kan skala bort alla onödiga brancher, minska
koden och oftast fortsätta optimera en bit in det nya uttrycket.

\textbf{Irr} $\Phi$, kortform för Irreducible. Anropas när Omega eller Psi inte längre
kan optimeringa vidare. Irr kan då utifrån nuvarande continuations
bygga trädet eller byta continuation och fortsätta optimera på en
andra delar i uttrycket. Det är här vi kan tillslut gå tillbaka till
STG-maskinen.

\begin{align*}
\IRR{\oLet xes}H{e'} & \Rightarrow & \OMEGA{\oLetObj x{\THUNK{e'}}s}He\\
\IRR{\oCase{brs}s}He & \Rightarrow & \IRR sH{\eCase e brs}
\end{align*}


Istället för att skapa en $\eCase e brs$ har vi även testat att gå in
och optimera varje branch, detta brytter dock semantiken, vi kan fastna
i oändliga loopar i brancher vi aldrig hade hamnat i. 

\begin{codeEx}
case x of
    { True  -> 5 : Nil
    ; False -> repeat 1
    }
\end{codeEx}
Om x är okänd och vi försöker optimera False branchen kommer vi fastna
i en oändlig loop, dock är det möjligt att när programet körs kommer
x bara anta värdet True. 

\begin{align*}
\IRR{\oLetObj x{OBJ}s}He & \Rightarrow & \IRR sH{\eLet x{OBJ}e}\\
\IRR{\cOFUN{\bar{x}}{\alpha}s}He & \Rightarrow & \mc{\alpha}s{\heap{\alpha}{\FUN{\bar{x}}e}}
\end{align*}




\textbf{STG}, då vi interagerar med STG maskinen måste vi har tydliga regler
för när maskien ska sluta evaluera och börja optimera. Detta sker
antingen då den stöter på en $\OPT t$ för första gången, eller när den
inte längre kan evaluera ett uttryck och har en Optimise continuation
på stacken.

\begin{align*}
\mc as{\heap a{\OPT t}} & \Rightarrow\\
\mc t{\cOPT{a\,}s}{\heap a{\BH}}\\
\mc x{\cOPT as}{\heapp x{\PAP f{a_{1}\cdots a_{n}}}f{\FUN{x_{1}\cdots x_{m}}e}} & \Rightarrow\\
\OMEGA{\cOFUN{x_{n+1}\cdots x_{m}}as}H{e[a_{1}/x_{1}\cdots a_{n}/x_{n}]}\\
\mc v{O.\star\,:\, s}{\nheap v{\THUNK e}} & \Rightarrow\\
\PSI{O.\star\,:\, s}Hv
\end{align*}

\subsubsection{Problem med CBV semantiken}

Oturligt nog bröt ovanstående optimise semantik mot STGs egna semantik. 
Som bekant forceras evaluering av thunkar endast i case-granskaren. 
Om funktionen som optimeras hade forcerat en thunk under vanlig körning så är
 opimise också tillåten att göra detta. Det finns ibland anledning att focera
 även vid andra tillfällen, men vårt naiva tillvägagångsätt höll inte måttet.

\begin{codeEx}
take n list = if (n == 0) Nil (head list : take (n-1) (tail list))
\end{codeEx}
Här vill man först evaluera n == 0 innan någon av argumenten till if.
If är defienierad med en case sats på det första argumentet. Även om
if inlinas så ligger inte casen-först i funktionen. För att bättre förstå varför kan vi observera hur take koden ser ut osockrad. 

\begin{codeEx}
take n list = let 
    { t1 = THUNK (n == 0)
    ; t2 = THUNK (head list)
    ; t3 = THUNK (n - 1)
    ; t4 = THUNK (tail list)
    ; t5 = THUNK (take t3 t4)
    } in  if t1 Nil t5;
\end{codeEx}

Vi evaluerar varje ny thunk efter vi passerat den, när vi kommer till 
\miniCode{t5} görs ett rekursivt anrop till take och processen börjar om.
\miniCode{if t1 Nil t5} får aldrig en chans att avbryta loopen och optimeringen terminerar aldrig.

Definierades take istället på följande sättet terminerar optimise:

\begin{codeEx}
take n list = case n == 0 of
    { True -> Nil
    ; False -> let 
        { t2 = THUNK (head list)
        ; t3 = THUNK (n - 1)
        ; t4 = THUNK (tail list)
        } in  take t3 t4
    };
\end{codeEx}

\NOTE{oeftersträvansvärt var ordet}

Och det är oeftersträvansvärt att tvinga användarna att skriva all funktioner
de vill optimera med detta i åtanke. Användaren behöver inte bara hålla reda på 
hur sina funktioner är skrivan, men också hur alla funktioner han använder är skrivna. 
Detta arbetar emot principen om abstraktion vilket är orimligt. Detta leder oss in på vår strävan
efter en lat optimeringssemantik.


\end{document}

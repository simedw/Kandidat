
\documentclass[../Optimise]{subfiles}
\begin{document}

\subsection{Introduktion till optimise}

Vi börjar med att återigen betrakta powerfunktionsexemplet från introduktionen. Vi är
intresserade av att se vilka optimeringar som krävs för att den skall transformeras till den
mer optimala kod som visades.
\begin{codeEx}
power n x = let t1 = n == 0
    in case t1 of
        { True -> 1
        ; False -> let 
            { t2 = n - 1 
            ; t3 = power t2 x
            ; t4 = x * t3
            } in t4
        };
\end{codeEx}
Vi ser att \ic{power 3 5} kommer att upphöja $5$ till $3$, dvs $5 * 5 * 5$.
Precis som innan gäller det att om vi känner till exponenten men inte basen kan vi göra en specialiserad version av \ic{power}, vi ska nu visa hur detta går till på ett informellt sätt. 
%Om vi känner till exponenten men inte basen kan vi göra en specialiserad version av \ic{power}
%för just den exponenten. Hur detta går till ska vi nu visa på ett informellt sätt. 
Senare kommer mer formella regler att ges. Vi antar att programmeraren
har skrivit följande för att visa att hen vill ha en specialiserad version av \ic{power} för $3$.

\begin{codeEx}
optimise (power 3)
\end{codeEx}

Vi börjar med att byta ut alla \ic{n} mot $3$ i \ic{power}. Ingenting görs med 
\ic{x} eftersom den är okänd. Vi ger också ett nytt namn till den optimerade funktionen
så att vi kan behålla originalet om den skulle behövas i någon annan del av koden.
\begin{codeEx}
power_3 x = let t1 = 3 == 0
    in case t1 of
        { True -> 1
        ; False -> let 
            { t2 = 3 - 1 
            ; t3 = power t2 x
            ; t4 = x * t3
            } in t4
        };
\end{codeEx}

Vi börjar nu gå igenom trädet. Först möter vi \ic{let t1 = ... in exp} vilket 
sparas undan och vi går vidare in i \ic{exp} som är ett \ic{case}-uttryck. 
\ic{Case}-uttrycket kan få ut värdet från \ic{t1} eftersom den inte innehåller några 
fria variabler (som i det här fallet bara är \ic{x}). Då \ic{3 == 0} är falskt fås nu: 

\begin{codeExDiff}
power_3 x = case |False| of
    { True -> 1
    ; False -> let 
        { t2 = 3 - 1 
        ; t3 = power t2 x
        ; t4 = x * t3
        } in t4
    };
\end{codeExDiff}

Nu kan vi välja \ic{False}-grenen och strunta i den andra. När vi går in i grenen
sparar vi som tidigare först undan \ic{let}:arna och går sedan till \ic{t4}.
Den kan bytas ut mot sitt värde i \ic{let}-uttrycket, och på samma sätt  kan \ic{t3}
och \ic{t2} infogas så att vi till slut kommer fram till följande funktion:

\begin{codeEx}
power_3 x = x * power (3-1) x;
\end{codeEx}

Härifrån kan vi fortsätta genom att infoga \ic{power (3-1) x} och optimera den
koden på samma sätt. Till sist får vi följande kod:

\begin{codeEx}
power_3 x = x * x * x * 1;
\end{codeEx}

Denna funktion är mycket mindre och snabbare.
Detta är dock inte riktigt hela sanningen eftersom heltal boxas i vårt språk, vilket
också betyder att multiplikationsoperatorn, \ic{*}, är en funktion som först
evaluerar sina argument och sedan multiplicerar dem med den primitiva 
multiplikationsoperatorn (\ic{*\#}), se sektion \ref{sec:boxing}.
Om vi nu infogar alla multiplikationer i \ic{power 3} får vi följande mastiga kod:
\begin{codeEx}
power_3 x = case 
    ( case 
        ( case 
            ( case x of
                { I# a -> case x of
                    { I# b -> case a #* b of
                        { r' -> let r = I# r' in r}}}
            ) of
            { I# c -> case x of
                { I# d -> case c #* d of
                    { r' -> let r = I# r' in r}}}                
         ) of
         { I# e -> case x of
              { I# f -> case e #* f of
                   { r' -> let r = I# r' in r}}}
    ) of
        { I# g -> case g #* 1 of
            { r' -> let r = I# r' in r}};                                     
\end{codeEx}


Vi \ic{case}:ar flera gånger på \ic{x} och använder konstruktorn \ic{I\#} för att boxa värden
när de ändå bara ska \ic{case}:as fram.
Detta kan vi optimera bort, vilket beskrivs i avsnittet om afterburner, och vi får följande funktion:

\begin{codeEx}
power_3 x = case x of
    { I# x' -> case x' #* 1 of
        { a -> case x' #* a of
            { b -> case x' #* b of
                { c -> case x' #* c of
                    { r' -> let r = I# r' in r}}}}};
\end{codeEx}

Detta är så långt som vår optimering kommer. \ic{x' *\# 1} skulle naturligtvis
kunna optimeras till \ic{x'}, men vi har inte fokuserat på sådana aritmetiska
specialfall. I resten av det här avsnittet kommer vi i detalj förklara vart och
ett av optimeringsstegen och varför de fungerar som de gör.

\end{document}

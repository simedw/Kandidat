
\documentclass[../Optimise]{subfiles}
\begin{document}

\section{Introduktion till optimise}

Vi börjar med att återigen betrakta powerfunktionsexemplet från introduktionen. Det som är
intressant är att se vilka optimeringar som krävs för att den skall transformeras till den
mer optimala kod som visades. Här kommer funktionen:

\begin{codeEx}
power n x = let t1 = n == 0
    in case t1 of
        { True -> 1
        ; False -> let 
            { t2 = n - 1 
            ; t3 = power t2 x
            ; t4 = x * t3
            } in t4
        };
\end{codeEx}

Om funktionen appliceras på $3$ och $5$, alltså \ic{power 3 5}, kommer den att upphöja $5$ till $3$, dvs $5 * 5 * 5$.
Precis som innan gäller det att om exponenten är känd men inte basen kan en 
specialiserad version av \ic{power} skapas. Detta visas först på ett informellt sätt, 
och senare kommer mer formella regler att ges. Anta att programmeraren
har skrivit följande för att visa att hon vill ha en specialiserad version av \ic{power} för $3$.

\begin{codeEx}
optimise (power 3)
\end{codeEx}

Först byts alla \ic{n} ut mot $3$ i funktionskroppen. Ingenting görs med 
\ic{x} eftersom den är okänd. Den optimerade funktionen får också ett nytt namn
så att originalet finns kvar om det skulle behövas i någon annan del av koden.

\begin{codeEx}
power_3 x = let t1 = 3 == 0
    in case t1 of
        { True -> 1
        ; False -> let 
            { t2 = 3 - 1 
            ; t3 = power t2 x
            ; t4 = x * t3
            } in t4
        };
\end{codeEx}

Nu traverseras trädet. Först kommer \ic{let t1 = ... in exp} som 
sparas undan så att \ic{exp}, som är ett \ic{case}-uttryck, kan behandlas. 
\ic{Case}-uttrycket kan få ut värdet från \ic{t1} eftersom \ic{t1} inte innehåller några 
fria variabler (i det här fallet är bara \ic{x} en fri variabel). Då \ic{3 == 0} är falskt fås nu: 

\begin{codeExDiff}
power_3 x = case |False| of
    { True -> 1
    ; False -> let 
        { t2 = 3 - 1 
        ; t3 = power t2 x
        ; t4 = x * t3
        } in t4
    };
\end{codeExDiff}

Nu kan \ic{False}-grenen väljas och den andra kan tas bort. När grenen traverseras
sparas först \ic{let}:arna undan och optimeringen kommer till \ic{t4} som kommer
efter \ic{let}-bindningarna.
\ic{t4} kan bytas ut mot sitt värde i \ic{let}-uttrycket, och på samma sätt  kan \ic{t3}
och \ic{t2} infogas så att den slutliga funktionen blir följande:

\begin{codeEx}
power_3 x = x * power (3-1) x;
\end{codeEx}

Härifrån fortsätter optimeringen genom att infoga \ic{power (3-1) x} och optimera den
koden på samma sätt. Till sist erhålls följande kod:

\begin{codeEx}
power_3 x = x * x * x * 1;
\end{codeEx}

Denna funktion är mycket mindre och snabbare.
Men i STG-språket boxas egentligen heltal, vilket
också betyder att multiplikationsoperatorn, \ic{*}, är en funktion som först
evaluerar sina argument och sedan multiplicerar dem med den primitiva 
multiplikationsoperatorn \ic{*\#} (se sektion \ref{sec:boxing} om boxing).
Om nu alla multiplikationer i \ic{power 3} infogas som de ser ut i STG 
fås egentligen följande mastiga kod:

\begin{codeEx}
power_3 x = case 
    ( case 
        ( case 
            ( case x of
                { I# x.a -> case x of
                    { I# x.b -> case x.a #* x.b of
                        { r' -> let r = I# r' in r}}}
            ) of
            { I# c -> case x of
                { I# x.c -> case c #* x.c of
                    { r' -> let r = I# r' in r}}}                
         ) of
         { I# d -> case x of
              { I# x.d -> case d #* x.d of
                   { r' -> let r = I# r' in r}}}
    ) of
        { I# e -> case e #* 1 of
            { r' -> let r = I# r' in r}};                                     
\end{codeEx}

Här kan vi använda en identitet för \kw{case}-satser som beskrivs i 
sektion \ref{sec:CaseLaw}, som gör att koden ser ut så här:

\begin{codeEx}
power_3 x = case x of
    { I# x.a -> case x of
        { I# x.b -> case x of
            { I# x.c -> case x of
                { I# x.d -> case x.a #* 1 of
                    { a -> case x.b #* a of
                        { b -> case x.c #* b of
                            { c -> case x.d #* c of
                                { r' -> let r = I# r' in r}}}}};
\end{codeEx}

Detta avslöjar att här görs \ic{case} flera gånger på \ic{x} och 
konstruktorn \ic{I\#} används för att boxa värden
när de ändå bara ska \ic{case}:as fram.
Detta kan optimeras bort, vilket beskrivs i avsnittet om afterburner
(sektion \ref{sec:Afterburner}) och vi får följande funktion:

\begin{codeEx}
power_3 x = case x of
    { I# x' -> case x' #* 1 of
        { a -> case x' #* a of
            { b -> case x' #* b of
                { c -> case x' #* c of
                    { r' -> let r = I# r' in r}}}}};
\end{codeEx}

Detta är så långt som optimeringen kommer. \ic{x' *\# 1} skulle naturligtvis
kunna optimeras till \ic{x'}, men i det här projektet ligger inte fokus på sådana 
aritmetiska identiteter eller andra liknande
specialfall. I resten av det här avsnittet förklaras i detalj alla
optimeringssteg och varför de fungerar som de gör.

\end{document}

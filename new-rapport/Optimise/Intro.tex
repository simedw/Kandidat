
\documentclass[../Optimise]{subfiles}
\begin{document}

\subsection{Introduktion till optimise}

Vi börjar med att återigen betrakta powerfunktionsexemplet från introduktionen.

\begin{codeEx}
power n x = let t1 = Thunk (n==0)
              in case t1 of
                      True -> 1
                      False -> let t2 = n - 1 
                                   t3 = power t2 x
                                   t4 = x * t3
                                in t4
\end{codeEx}
Vi ser att \miniCode{power 3 5} kommer upphöja 5 till 3, dvs 5 * 5 * 5

Om vi känner till exponenten men inte basen kan vi göra en optimerad version av power
specialicerad för just den exponenten, hur detta går till ska vi nu försöka visa på ett 
informellt sätt. Misströsta ej alla detaljer ges senare.

\begin{codeEx}
optimise (power 3)
\end{codeEx}

Vi börjar med att byta ut alla n mot 3 i power. Ingeting kan göras med \miniCode{x} då 
den är okänd. Vi `döper' också om power för att inte förstöra orginalet.
\begin{codeEx}
power_3 x = let t1 = Thunk (3==0)
               in case t1 of
                     True -> 1
                     False -> let t2 = 3 - 1 
                                  t3 = power t2 x
                                  t4 = x * t3
                               in t4
\end{codeEx}

Vi börjar nu gå igenom trädet. Först möter vi \miniCode{let t1 = ... in exp} vilket sparas undan och vi gå vidare in i \miniCode{exp} som är ett case-uttryck. Case:en kan forcera fram värdet oss t1 då t1 är fri från fria variabler. Då \miniCode{3==0} är falskt fås nu: 

\begin{codeEx}
power_3 x = case False of
              True -> 1
              False -> let t2 = 3 - 1 
                           t3 = power t2 x
                           t4 = x * t3
                         in t4
\end{codeEx}

Vi kan nu välja False branchen, vi sparar sedan undan let:arna och går in i t4. Vi inlinar det hela
(Egentligen allokeras 3-1 på heapen och forceras fram senare)

\begin{codeEx}
power_3 x = x * power (3-1) x
\end{codeEx}

Vi forstätter med att inline power (3-1) x på samma sätt.

\begin{codeEx}
power_3 x = x * x * x * 1
\end{codeEx}

Vilket är en mycket mindre och snabbare funktion.
Men det här är inte hela sanningen, egentlingen boxas varje heltal i vårt språk 
(står säkert någon annanstans). Och + är som bekant bara socker för att case 
fram värdet i av intarna och sedan addera det med en primitiv plus operator \#+

\begin{codeEx}
a * b
\end{codeEx}

blir efter inlining

\begin{codeEx}
case a of
    I# a' -> case b of
        I# b' -> case  a' #* b' of
            r' -> let r = I# r'
                      in r
\end{codeEx}

Och om vi då inlinarna alla multiplicationer i power\_3 får vi följande kod:
\begin{codeEx}
power_3 x = case (
                     case (
                     case x of
                        I# a -> case x of
                            I# b -> case a #* b of
                                  r' -> let r = I# r'
                                   in r
                        )
                          of
                        I# c -> case x of
                            I# d -> case c #* d of
                                  r' -> let r = I# r'
                                   in r
                   case x of
                        I# a -> case x of
                            I# b -> case a #* b of
                                  r' -> let r = I# r'
                                   in r
                        )
                          of
                        I# c -> case c #* 1 of
                                  r' -> let r = I# r'
                                   in r                                                         
\end{codeEx}


Vi case:ar flera gånger på x och bygger upp I\# för att direkt ta bort de senare.
 Detta kan naturligtvis också optimeras.

\begin{codeEx}
power_3 x = case x of
               I# x' -> case x' #* x' of
                       a -> case x' #* a of
                          b -> case x' #* b of
                             c -> case x' #* 1 of
                               r' -> let r = I# r'
                                 in r
\end{codeEx}

Detta är hur långt vår optimise funktionalitet kommer. x * 1 skulle naturligtvis
kunna optimeras till x, men vi har inte fokuserat på dessa corner cases.
I resten av det här avsnittet kommer vi i detalj förklara vart och ett av dessa 
steg och varför det fungerar som det gör.

\end{document}

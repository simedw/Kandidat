
\documentclass[../Optimise]{subfiles}
\begin{document}

\subsection{Introduktion till optimise}

Vi börjar med att återigen betrakta powerfunktionsexemplet från introduktionen. Vi är
intreserade av att se vilka optimeringar som krävs för att den skall optimeras till den
mer optimala kod som visades.
\begin{codeEx}
power n x = let t1 = n==0
    in case t1 of
        { True -> 1
        ; False -> let 
            { t2 = n - 1 
            ; t3 = power t2 x
            ; t4 = x * t3
            } in t4
        };
\end{codeEx}
Vi ser att \miniCode{power 3 5} kommer upphöja 5 till 3, dvs 5 * 5 * 5

Om vi känner till exponenten men inte basen kan vi göra en optimerad version av power
specialicerad för just den exponenten, hur detta går till skall vi nu visa på ett 
mer informellt sätt. Senare kommer mer formella regler att ges. Vi antar att programmeraren
har skrivit följande för att visa att hon vill ha en specialiserad version av \miniCode{power} för $3$.

\begin{codeEx}
optimise (power 3)
\end{codeEx}

Vi börjar med att byta ut alla n mot 3 i power, ingeting görs med \miniCode{x} för 
att den är okänd. Vi `döper' också om power för att inte förstöra orginalet.
\begin{codeEx}
power_3 x = let t1 = 3==0
    in case t1 of
        { True -> 1
        ; False -> let 
            { t2 = 3 - 1 
            ; t3 = power t2 x
            ; t4 = x * t3
            } in t4
        };
\end{codeEx}

Vi börjar nu gå igenom trädet. Först möter vi \miniCode{let t1 = ... in exp} vilket sparas undan och vi gå vidare in i \miniCode{exp} som är ett case-uttryck. Case:en kan forcera fram värdet från t1 då t1 är inte innehåller några fria variabler, som i det här fallet är x. Då \miniCode{3==0} är falskt fås nu: 

\begin{codeEx}
power_3 x = case False of
    { True -> 1
    ; False -> let 
        { t2 = 3 - 1 
        ; t3 = power t2 x
        ; t4 = x * t3
        } in t4
    };
\end{codeEx}

Vi kan nu välja False grenen, vi sparar sedan undan let:arna och går in i t4, därefter inlinas t4.

\begin{codeEx}
power_3 x = x * power (3-1) x;
\end{codeEx}

Vi forstätter med att inline power (3-1) x på samma sätt.

\begin{codeEx}
power_3 x = x * x * x * 1;
\end{codeEx}

Vilket är en mycket mindre och snabbare funktion.
Men det här är inte hela sanningen, egentlingen boxas varje heltal i vårt språk.
 Och * är som bekant en funktion som evaluerar dess argument och sedan multiplicerar
dem med den primitiva gånger operatorn *\#

\begin{codeEx}
a * b
\end{codeEx}

blir efter inlining

\begin{codeEx}
case a of
    { I# a' -> case b of
        { I# b' -> case  a' #* b' of
            { r' -> let r = I# r' in r}}}
\end{codeEx}

Och om vi då inlinarna alla multiplicationer i power\_3 får vi följande svårbegriblig kod:
\begin{codeEx}
power_3 x = case 
    ( case 
        ( case 
            ( case x of
                { I# a -> case x of
                    { I# b -> case a #* b of
                        { r' -> let r = I# r' in r}}}
            ) of
            { I# c -> case x of
                { I# d -> case c #* d of
                    { r' -> let r = I# r' in r}}}                
         ) of
         { I# a -> case x of
              { I# b -> case a #* b of
                   { r' -> let r = I# r' in r}}}
    ) of
        { I# c -> case c #* 1 of
            { r' -> let r = I# r' in r}};                                     
\end{codeEx}


Vi case:ar flera gånger på x och bygger upp I\# för att direkt ta bort de senare.
 Detta kommer vi att optimera bort och få följande funktion.

\begin{codeEx}
power_3 x = case x of
    { I# x' -> case x' #* x' of
        { a -> case x' #* a of
            { b -> case x' #* b of
                { c -> case x' #* 1 of
                    { r' -> let r = I# r' in r}}}}};
\end{codeEx}

Detta är hur långt vår optimise funktionalitet kommer. \miniCode{x * 1} skulle naturligtvis
kunna optimeras till x, men vi har inte fokuserat på dessa specialfall.
I resten av det här avsnittet kommer vi i detalj förklara vart och ett av dessa 
steg och varför det fungerar som det gör.

\end{document}

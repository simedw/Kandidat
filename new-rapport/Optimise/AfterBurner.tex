\documentclass[../Optimise]{subfiles}
\begin{document}

\section{Afterburner}
\label{sec:Afterburner}

Alla optimeringar hittills i det här arbetet har arbetat tillsammans, i en gemensam
traversering av syntaxträdet som följer en gemensam semantik för att nå det önskade resultatet. 
Detta är bra ur tidssynpunkt men gör det ibland svårt att byta ut enskilda delar i koden
eftersom många ställen då ofta behöver ändras på samma gång.

Det finns även en del optimeringar som behöver se större delar av syntaxträdet
samtidigt för att kunna vara verksamma. Dessa optimeringar är som regel långsammare att utföra.

Vi har valt att utföra ett antal sådana optimeringar efter att de vanliga
optimeringana har körts, därav namnet `after burner'. Följande regel beskriver
var i semantiken som afterburnern kommer in:

\begin{align*}
\CBNIRR{\cOFUN{\many{a}}{\alpha}s}HAe \Rightarrow & \CBNMC{\alpha}{s}{\heap{\alpha}{obj}}A \\
 & \text{där } obj = \text{afterburn} (\FUN{\many{a}}e)
\end{align*}

\paragraph{Gemensam case-granskare} 
Att \kw{case}:a på samma uttryck två gånger kommer i ett funktionellt språk alltid att ge
samma resultat då alla variabler egentligen är konstanter. För varje \kw{case}-gren
som optimeringen går in i är det säkert att alla nästlade \kw{case}:ar på samma uttryck måste välja samma
gren som den första.

\begin{codeEx}
test x = x * x;
\end{codeEx}

Blir efter vanlig optimering

\begin{codeEx}
test x = case x of
            { I# a -> case x of
                { I# b -> case a #* b of
                    { r' -> let r = CON (I# r')
                            in r
            }   }   };
\end{codeEx}

Om optimeringen för gemensamma \kw{case}-granskare sedan körs erhålls följande:
\begin{codeEx}
test x = case x of
            { I# a -> case a #* a of
                { r' -> let r = CON (I# r')
                        in r
            }   };
\end{codeEx}

Generellt sett kan regeln uttryckas som följande:
\begin{codeEx}
case x of
    C a b -> e1
    D a b -> e2
\end{codeEx}

\ic{case x} söks upp i \ic{e1} och 
\ic{C}-grenen initieras med \ic{a} och \ic{b} som värden. I \ic{e2} görs samma 
sak men i det fallet är det \ic{D}-grenen som initieras.


\end{document}

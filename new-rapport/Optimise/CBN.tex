\documentclass[../Optimise]{subfiles}
\begin{document}

\numberwithin{equation}{section}

\global\long\def\subst#1#2#3{#1[#2/#3]}
\global\long\def\abyss#1#2{A[#1\mapsto#2]}

\global\long\def\cUpd#1#2{Upd\,#1\,:\,#2}
\global\long\def\cOInstant#1#2{OInstant\,#1\,:\,#2}
\global\long\def\cOApp#1#2#3{OApp\,#1\;#2\,:\,#3}
\global\long\def\cOUpd#1#2{OUpd\,#1\,:\,#2}


\global\long\def\objCon#1{CON(#1)}
\global\long\def\eLet#1#2#3{\mathtt{let}\,#1=#2\,\mathtt{in}\,#3}

\global\long\def\Arg#1#2{Arg\,#1\,:\,#2}



\subsection{Call-by-name-semantik}
För att lösa problemet med call-by-value-semantiken så har vi skapat en latare semantik.
Det är en svår balansgång mellan att optimera för mycket och för lite, men vi har
försökt lösa lathetsproblemet på samma sätt som STG själv gör det.
\subsubsection{Abyss}
Det stora problemet med den förgående semantiken var thunkar optimerades där
de definerades och och inte där de används, om de används, vilket kunnde leda
till onödig optimering eller till och med icke-terminering!

Vår lösning går ut på att let-uttryck allokerar sina objekt på heapen även när vi
optimerar. Men för att kunna skilja på objekt som kan innehålla okända variabler
och de vanliga heap objekten som aldrig innehåller något okänt, har vi introducerat
en ny slags heap som vi kallar för `Abyss'. I reglerna kommer denna att denoteras
med stort A, och användas på liknande sätt som Heapen H.


\subsubsection{Funktionsapplikationer}
\begin{align}
\CBNOMEGA s{\heap f{\FUN{x_{1}\ldots x_{n}}e}}A{f\, a_{1}\cdots a_{n}} \Rightarrow & \CBNMC f{\Arg{a_{1}}{\ldots\,:\,\Arg{a_{n}}{\cOInstant 1s}}}HA
\end{align}

Den här regeln används för inlining, vi låter maskinen få göra det då vi lägger ut en
\cont{OInstant} som tillåter maskinen att reducera $1$ steg. Efter det kommer optimeringen
fortsätta. Vad som inte syns i denna formulering är att vi endast gör det om
inline-räknaren tillåter det.

\NOTE{Detta är en ful regel, jepp}

\begin{align}
\CBNOMEGA s{\heap f{\THUNK e}}A{f\, a_{1}\ldots a_{n}}  \Rightarrow & \CBNMC f{\cOApp f{a_{1}\ldots a_{n}}s}HA\\
\CBNOMEGA sH{\abyss f{\THUNK e}}{f\, a_{1}\ldots a_{n}}  \Rightarrow & \CBNOMEGA{\cOApp f{a_{1}\ldots a_{n}}s}HAe\\
\label{CBN:Fun:Irr}\CBNIRR{\cOApp f {a_{1}\ldots a_{n}}s}HAe  \Rightarrow & \CBNIRR sHA{f\, a_{1}\ldots a_{n}}\\
\CBNPSI{\cOApp f{a_{1}\ldots a_{n}}s}HA{lbs}v  \Rightarrow & \CBNOMEGA sHA{v\, a_{1}\ldots a_{n}\footnote{ with the let bindings from \ensuremath{lbs} added}}
\end{align}

Vi introducerar en ny continuation \cont{OApp} som pekar på ett objekt som är av
funktionstyp, samt har en samling av argument. Den används för att funktionen är
ännu inte evaluerad till ett funktions objekt, och därför behöver vi evaluera det.
Om det inte lyckades \eqref{CBN:Fun:Irr} så använder vi det gamla funktionsvärdet
när vi bygger upp det igen.

\begin{align}
\CBNOMEGA sHA{\oplus\, a_{1}\ldots a_{n}}  \Rightarrow & \mc{\oplus\, a_{1}\ldots a_{n}}{\cOInstant 2s}HA
\end{align}

Värt att notera är att primitiva operationer så som \miniCode{+\# *\# ==\#} betecknas $\oplus$. 
Dessa körs också via maskinen, så först måste man även kontrollera att dess argument
är kända.

\subsubsection{let-uttryck}
\begin{align}
\label{CBN:Let1} \CBNOMEGA s{\heap{a_{1}\cdots a_{n}}{h_{1}\cdots h_{n}}}A{\eLet x{\objCon{C\, a_{1}\cdots a_{n}}}r} \\ \nonumber \Rightarrow & \CBNOMEGA s{\heap{x'}{C\, a_{1}\cdots a_{n}}}A{\subst r{x'}x}\\
   & x'\mbox{ fresh} \nonumber \\
\label{CBN:Let2} \CBNOMEGA sHA{\eLet x{obj}r} \\ \nonumber \Rightarrow & \CBNOMEGA{\oLet{x'}s}H{\abyss{x'}{obj}}{\subst r{x'}x}\\
   & x'\mbox{ fresh} \nonumber \\
\label{CBN:Let3} \CBNIRR{\oLet xs}H{\abyss xo}e \Rightarrow & \CBNIRR sHA{\eLet xoe}\\
\label{CBN:Let4} \CBNIRR{\cOUpd vs}HAe \Rightarrow & \CBNIRR sH{\abyss v{\THUNK e}}e \\
\label{CBN:Let5} \CBNPSI{\oLet xs}H A{lbs}v \Rightarrow & \CBNPSI sHA{x\,:\, lbs}v \\
\label{CBN:Let6} \CBNPSI{\cOUpd xs}{\heap vo}A{lbs}v \Rightarrow & \CBNPSI sH{\abyss xo}{lbs}v\\
\CBNPSI{\cOUpd xs}H{\abyss vo}{lbs}v \Rightarrow & \CBNPSI sH{\abyss xo}{lbs}v\\
\label{CBN:Let7} \CBNPSI{\cUpd xs}{\heap vo}A{lbs}v  \Rightarrow & \CBNPSI sH{\abyss xo}{lbs}v 
\end{align}

Alla \miniCode{let} som $\Omega$ ser kommer nu att läggas på abyssen, med undantag av \miniCode{CON} om alla varaibler kända \eqref{CBN:Let2}. Detta undantag infördes då det underlätta optimering i case:ar, vi kan inte case på något på abyssen.

$\Psi$ kan ändast få variabler som argument, aldrig uttryck. Detta innebär att vi i regler \eqref{CBN:Let5} vet med säkerhet att variablen inte refererar till det let-bunda uttrycket. Vi kan därför ta bort det från stacken och fortsätta vår optimering. I $\Phi$ får vi dock ett uttryck som argument och måste därför återbygga let-satsen. 

De resterade reglerna behandlar \cont{OUpd} vilket läggs ut när innehållet i en thunk optimeras. 

%När vi slutligen når $\Psi$ med en OLet-continuation \eqref{CBN:Let5} på stacken kan vi vara säkra på vi inte använder den bortsparade let-satsen och vi kan därför kasta bort den. I \eqref{CBN:Let3} kan dock uttrycket innehålla en reference till let:en och därför återskapas uttrycket. 



\subsubsection{case-satser}
Fungerar precis som tidigare, dock forslas let bunda variabler samt abyssen nu också runt.
%\begin{align}
%\label{eq:case1}\CBNOMEGA sHA{\eCase e{brs}}  \Rightarrow &\, \CBNOMEGA{\oCase{brs}s}HAe\\
%\label{eq:case2}\CBNIRR{\oCase{brs}s}HAe  \Rightarrow &\, \CBNIRR sHA{\eCase e{brs}}\\
%\label{eq:case3}\CBNPSI{\oCase{brs}s}HA{lbs}v  \Rightarrow &\, \CBNOMEGA sHA{e'}
%\end{align}
%Där \miniCode{e'} är instantierad med den korrekta grenen och let bundna variabler från \miniCode{lbs} läggs till.
%Vi ser att \eqref{eq:case1} är oförändrade från CBV-semantiken. 

\subsubsection{Atoms}
\begin{align}
\label{CBN:Atoms1}\CBNOMEGA s{\heap v{\THUNK e}}Av \Rightarrow &\, \CBNMC vsHA\\
\label{CBN:Atoms2}\CBNOMEGA sH{\abyss v{\THUNK e}}v  \Rightarrow &\, \CBNOMEGA{\cOUpd vs}HAe\\
\label{CBN:Atoms3}\CBNOMEGA sH{\abyss v{CON(C\, a_{1}\cdots a_{n})}}v \Rightarrow &\, \CBNPSI sHA{\epsilon}v \\
\label{CBN:Atoms4}\CBNOMEGA s{\heap v{\objCon{C\, a_{1}\cdots a_{n}}}}Av \Rightarrow &\, \CBNPSI sHA{\epsilon}v
\end{align}
Regel \eqref{CBN:Atoms1} så lämnar vi arbetet vidare åt maskinen, den får fortsätta att
evaluera thunken som den fick in.

Liknande behövs göras ifall thunken är på abyssen, dock kan vi inte använda maskinen
så vi får kalla vidare med $\Omega$, vilket görs av regel \eqref{CBN:Atoms2}. När vi 
sedan kommer bygga upp let-uttrycket som bundit denna thunk vill vi där att man skall
använda det optimerade värdet av thunken. Därför lägger vi ut en \cont{OUPd}
continuation som kommer att uppdatera med ett förhopningsvis optimerat resultat.

De övriga reglerna är att vi har hittat ett värde och $\Psi$ funktionen kan kallas.
Vilket kommer leda till att dessa kan användas i tidigare case-utryck eller liknande.


%\subsubsection{Övrigt}

%\subsubsection{CBNOMEGA $\CBNOMEGA$}

%\subsubsubsection{Funtionsapplikationer}
%\subsubsubsection{let-applikationer}
%\subsubsubsection{Atoms}


%\subsubsection{CBNPSI $\CBNPSI$}

%\subsubsection{CBNIRR $\Phi$}


\end{document}
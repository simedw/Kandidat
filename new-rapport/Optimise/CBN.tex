\documentclass[../Optimise]{subfiles}
\begin{document}

\numberwithin{equation}{section}

\global\long\def\subst#1#2#3{#1[#2/#3]}
\global\long\def\abyss#1#2{A[#1\mapsto#2]}

\global\long\def\cUpd#1#2{Upd\,#1\,:\,#2}
\global\long\def\cOInstant#1#2{OInstant\,#1\,:\,#2}
\global\long\def\cOApp#1#2#3{OApp\,#1\;#2\,:\,#3}
\global\long\def\cOUpd#1#2{OUpd\,#1\,:\,#2}
\global\long\def\cOLet#1#2{OLet\,#1=\,?\,\mathtt{ in }\,\bullet\,:\,#2}

\global\long\def\objCon#1{CON(#1)}
\global\long\def\eLet#1#2#3{\mathtt{let}\,#1=#2\,\mathtt{in}\,#3}

\global\long\def\Arg#1#2{Arg\,#1\,:\,#2}



\section{Call-by-name-semantik}
\label{sec:Optimise:CBN}
%\setcounter{equation }{0}
För att lösa problemet med call-by-value-semantiken så har optimeringsfunktionens beteende ändrats så
att den följer en latare semantik. Dess regler, som här ska presenteras,
är en fortsättning på och förändning av reglerna från CBV-semantiken.
Deras mål är att följa den lata semantiken i STG, men fortfarande genomföra bra
optimeringar.
Det är en svår balansgång mellan att optimera för mycket och för lite, men genom
att följa semantiken för STG går det att uppnå lathet på samma sätt som STG själv gör det.

\paragraph{Abyss}
Det stora problemet med den föregående semantiken var att thunkar optimerades där
de definierades och och inte där de användes (om de användes), vilket kunde leda
till icke-terminering och i många fall onödig optimering.

Lösningen går ut på att \kw{let}-uttryck allokerar sina objekt på heapen även under
optimering, vilket inte görs i CBV-semantiken. Men för att kunna skilja på objekt som kan innehålla okända variabler
och de vanliga heap-objekten som aldrig innehåller något okänt, har 
en ny sorts heap, här kallad för `abyss', lagts till. Den består av objekt
som kan innehålla fria variabler, varför de inte ska blandas ihop med objekten 
på heapen som alla är helt kända.
I reglerna kommer abyssen att denoteras 
med stort $A$, och användas på liknande sätt som heapen $H$. Så en konfiguration
i tillståndet $\Omega$, skrivs nu:

\[
\CBNOMEGA SHAe
\]

\paragraph{}

Ett annat problem med den förra semantiken var att om det fanns ett \kw{let}-uttryck
i en granskare till en \kw{case}-sats, så kunde inget värde för granskaren fås fram.
För att ändra det beteendet har ytterligare en parameter lagts till konfigurationen.
Denna används endast av $\Psi$ för att hålla koll på alla \kw{let}-uttryck den har sett
som måste byggas upp igen senare. Eftersom den endast används av $\Psi$ kommer den bara
att skrivas ut för detta tillstånd, och det ser ut så här:

\[
\CBNPSI SHA{lbs}a
\]

\begin{comment}
En annan ändring är att $\Psi$ numera tar en atom istället för en variabel
eftersom det inte bara är variabler som kan vara värden. Detta var en detalj
som missades i CBV-semantiken men som nu har rättats till.
\end{comment}

\paragraph{Funktionsapplikationer}
Regeln \eqref{CBN:FunK} är samma som i den gamla semantiken och säger att om argumenten till funktionen $f$ är kända
så kan vi evaluera applikationen på maskinen. 

Hur infogningen görs beror på om implementationen använder anropsstack eller inte, och därför 
skriver regeln inte ut några detaljer om hur infogningen utförs, utan bara att den kommer att ske. 
Regel \eqref{CBN:FunI} visar att om parameterlistan för funktionen och argumentlistan för applikationen är
lika långa kommer funktionen att infogas med parametrarna $\many{x}$ utbytta mot
argumenten $\many{a}$. Detta görs bara om infogningsräknaren tillåter denna infogning.


\begin{align}
\label{CBN:FunK}\CBNOMEGA S{\heap f{\FUN{\many{x}}{e}}}A{f \, \many{a}} \rulesep & \CBNMC {f \, \many{a}}SHA \\
 & \many{a} \text{ kända } \nonumber \\
\label{CBN:FunI}\CBNOMEGA S{\heap f{\FUN{\many{x}}e}}A{f\, \many{a}} \rulesep & \CBNOMEGA SHA{\text{infoga } f \text{ med } \many{a}} \\
 & \text{ samma antal atomer i } \many{x} \text{ som i } \many{a} \nonumber
\end{align}

I de fall då variabeln inte är ett \obj{FUN}-objekt på heapen utan en thunk evalueras den först
till en funktion. En ny continuation \cont{OApp} införs för att maskinen
ska kunna återskapa funktionsapplikationen den arbetade med.

Om thunken kommer från abyssen kan inte den vanliga maskinen utnyttjas, utan optimeringen får
gå vidare in i $\Omega$-tillståndet:

\begin{align}
\CBNOMEGA S{\heap f{\THUNK e}}A{f\, \many{a}}  \rulesep & \CBNMC f{\cOApp f{\many{a}}S}HA\\
\CBNOMEGA SH{\abyss f{\THUNK e}}{f\, \many{a}}  \rulesep & \CBNOMEGA{\cOApp f{\many{a}}S}HAe
\end{align}

Om evalueringen lyckas få fram en funktion kommer \eqref{CBN:Fun:Psi} sedan att
på nytt bygga upp applikationen och fortsätta optimera funktionen i $\Omega$.
Notera att den måste lägga $lbs$ på stacken så att de kommer i samma ordning som när de togs.

Om evalueringen inte lyckades \eqref{CBN:Fun:Irr} används det gamla funktionsvärdet
när den byggs upp igen.

\begin{align}
\label{CBN:Fun:Psi}\CBNPSI{\cOApp f{\many{a}}S}HA{lbs}v  \rulesep & \CBNOMEGA {S'}HA{v\, \many{a}} \\
 & \text{där } S' \text{ är \kw{let}-bindningarna av } lbs \text{ tillagda till } S \nonumber \\
\label{CBN:Fun:Irr}\CBNIRR {\cOApp f {\many{a}}S}HAe  \rulesep & \CBNIRR SHA{f\, \many{a}}
\end{align}

Primitiva operationer som \ic{+\#}, \ic{*\#} och \ic{==\#} betecknas $\oplus$. 
Dessa körs också via maskinen, så även här måste dess argument kontrolleras så att de är kända.

\begin{align}
\CBNOMEGA SHA{\oplus\, \many{a}}  \rulesep & \CBNMC{\oplus\, \many{a}}SHA \\
 & \many{a} \text{ kända } \nonumber
\end{align}


\paragraph{Let-uttryck}
De \kw{let}-uttryck som bara har kända variabler kan allokeras på heapen\eqref{CBN:Let1} som vanligt
medans de med okända läggs på abyssen\eqref{CBN:Let2}.
Detta undantag görs för att få bättre delning, då thunkar på heapen alltid beräknas
av den vanliga maskinen och därför automatiskt får delning.

\begin{align}
\label{CBN:Let1} \CBNOMEGA SHA{\eLet x{obj}r} \rulesep & \CBNOMEGA {\cOLet{x'}S}{\heap{x'}{obj}}A{\subst r{x'}x}\\
   & x'\mbox{ ny variabel och } obj \text{ är känd}\nonumber \\
\label{CBN:Let2} \CBNOMEGA SHA{\eLet x{obj}r} \rulesep & \CBNOMEGA{\cOLet{x'}S}H{\abyss{x'}{obj}}{\subst r{x'}x}\\
   & x'\mbox{ ny variabel} \nonumber 
\end{align}

$\Psi$-tillståndet kommer i regel \eqref{CBN:Let3} att lägga till denna \kw{let}-binding
på sin lista över dessa.

 I $\Phi$ fås dock ett uttryck som argument och den måste därför 
återbygga \kw{let}-satsen, vilket sker via de två reglerna \eqref{CBN:Let4} och 
\eqref{CBN:Let5} beroende på om den ligger på heapen eller abyssen.

\begin{align}
\label{CBN:Let3} \CBNPSI{\cOLet xS}H A{lbs}v \rulesep & \CBNPSI SHA{x\,:\, lbs}v \\
\label{CBN:Let4} \CBNIRR{\cOLet xS}{\heap xo}Ae \rulesep & \CBNIRR SHA{\eLet xoe} \\
\label{CBN:Let5} \CBNIRR{\cOLet xS}H{\abyss xo}e \rulesep & \CBNIRR SHA{\eLet xoe}
% These rules are not needed 
%\label{CBN:Let4} \CBNIRR{\cOUpd vs}HAe \rulesep & \CBNIRR sH{\abyss v{\THUNK e}}e \\
%\label{CBN:Let6} \CBNPSI{\cOUpd xs}{\heap vo}A{lbs}v \rulesep & \CBNPSI sH{\abyss xo}{lbs}v\\
%\CBNPSI{\cOUpd xs}H{\abyss vo}{lbs}v \rulesep & \CBNPSI sH{\abyss xo}{lbs}v\\
%\label{CBN:Let7} \CBNPSI{\cUpd xs}{\heap vo}A{lbs}v  \rulesep & \CBNPSI sH{\abyss xo}{lbs}v \\ 
\end{align}


I avsnitt \ref{sec:DeadCode} ändras just \eqref{CBN:Let4} 
och \eqref{CBN:Let5} så att de endast bygger upp \kw{let}-uttrycket när så behövs.


%När vi slutligen når $\Psi$ med en OLet-continuation \eqref{CBN:Let5} på stacken kan vi vara säkra på vi inte använder den bortsparade let-satsen och vi kan därför kasta bort den. I \eqref{CBN:Let3} kan dock uttrycket innehålla en reference till let:en och därför återskapas uttrycket. 



\paragraph{Case-satser}
\kw{Case}-satser fungerar precis som tidigare, med den skillnaden att \kw{let}-bundna variabler 
och abyssen nu också forslas runt. Se sektion \ref{cbv:case}
%\begin{align}
%\label{eq:case1}\CBNOMEGA sHA{\eCase e{brs}}  \rulesep &\, \CBNOMEGA{\oCase{brs}s}HAe\\
%\label{eq:case2}\CBNIRR{\oCase{brs}s}HAe  \rulesep &\, \CBNIRR sHA{\eCase e{brs}}\\
%\label{eq:case3}\CBNPSI{\oCase{brs}s}HA{lbs}v  \rulesep &\, \CBNOMEGA sHA{e'}
%\end{align}
%Där \miniCode{e'} är instantierad med den korrekta grenen och let bundna variabler från \miniCode{lbs} läggs till.
%Vi ser att \eqref{eq:case1} är oförändrade från CBV-semantiken. 

\paragraph{Atomer}
Om en variabel pekar på en thunk i heapen kan den evalueras till ett
värde med den vanliga maskinen\eqref{CBN:Atoms1}. Ifall thunken var på abyssen
finns det fria variabler i den och då får optimeringen gå vidare med $\Omega$ \eqref{CBN:Atoms2}.

Alla andra atomer är värden, så då går optimeringen vidare till $\Psi$\eqref{CBN:Atoms3}.

\begin{align}
\label{CBN:Atoms1}\CBNOMEGA S{\heap v{\THUNK e}}Av \rulesep \CBNMC vSHA\\
\label{CBN:Atoms2}\CBNOMEGA SH{\abyss v{\THUNK e}}v  \rulesep \CBNOMEGA SHA{\text{infoga }e}\\
\label{CBN:Atoms3}\CBNOMEGA SHAa \rulesep &\, \CBNPSI SHA{\epsilon}a
%\label{CBN:Atoms4}\CBNOMEGA s{\heap v{\objCon{C\, a_{1}\cdots a_{n}}}}Av \rulesep &\, \CBNPSI sHA{\epsilon}v
\end{align}


\subsection{Call-by-name-semantik med anropsstack}
\label{sec:CBN:Callstack}

%\NOTE{Virrvarv av vad vi betyder}

De invarianter som måste hållas för att organisera en anropsstack liknar de som finns
när den organiseras för STG-maskinen. Här nedan listas de ställen i optimeringen där anropsstacken förändras:



\begin{itemize}
\item
Om $\Omega$ börjar optimera inuti en \kw{case}-granskare dupliceras översta aktiveringsposten 
på som i maskinen när granskaren börjar evalueras. När den är klar
tas den översta posten bort, och antingen kan \kw{case}-uttrycket byggas upp igen
eller rätt gren väljas. För att välja rätt gren ersätts alla argument som var
bundna via grenen med värdena som de ska ha. Detta sker via en specialsubstitution 
som både lägger in de värden som grenen band och skiftar ner grenuttryckets andra lokala 
variabler så många steg som antalet variabler som bands av grenen, så att uttryckets
lokala variabler får rätt värde även efter att \kw{case}-satsen tagits bort.

\item
När STG-maskinen anropas från optimeringen dupliceras den översta aktiveringsposten,
och när maskinen sedan har evaluerat fram ett värde och anropar $\Psi$ tas den översta aktiveringsposten bort igen.

\item
Precis som när \kw{case}-granskare optimeras dupliceras den översta aktiveringsposten när
funktionen i en funktionsapplikation optimeras. Detta beror på att att det likt \kw{case}
finns en punkt som optimeringen ska återvända till. Denna aktiveringspost tas bort när 
funktionsapplikationen byggs upp igen (i antingen $\Psi$ eller $\Phi$). 

\item
Varje \cont{OLet}-continuation behöver en plats i aktiveringsposten, så därför
läggs ett argument ut på aktiveringspsten när en \cont{OLet} läggs ut.
När en \cont{OLet} plockas bort tas på motsvarande sätt argumentet på aktiveringsposten bort.

\item
Om grenoptimeringar är aktiverade läggs variablerna som binds av grenen ut på
översta aktiveringsposten som okända, bara för att de ska ta upp rätt antal platser.
Dessa plockas sedan bord när optimeringen av grenen är klar. 


\end{itemize}

\end{document}

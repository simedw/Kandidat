
\documentclass[../Optimise]{subfiles}
\begin{document}

\subsection{Optimise ... with ..}

  Det är finns alltid specialfall där brott mot semantiken hade varit fördelaktigt.
  Sådana fall ytterst svåra att idenfiera maskinellt, därför ombes användaren
  ge ytterligare argument till optimeringsfunktionen när detta är dessa specialfall uppstår.

\begin{codeEx}
  optimise f x with { options }
\end{codeEx}



\subsubsection{Case Branches}
      Ibland finns det möjlighet att optimera i en casebranch. Detta bryter semantiken
      då detta kan forcera evaluering av uttryck som vanligtvis inte skulle beräknats.
\begin{codeEx}
f x y z = case g z of
        A -> case h x y of
              R -> t1 z
              S -> t2 z
        B -> case h y x of
              R -> t3 z
              S -> t4 z
\end{codeEx}


      Om vi nu använder \miniCode{optimise( f 1 2 )}, så är \miniCode{z} okänd, men i brancherna så
      finns det case-sastser som bara beror på \miniCode{x} och \miniCode{y} och rätt gren kan alltså väljas. Vi har stöd för att optimera i sådana här casegrenar:

\begin{codeEx}
  optimise f x y with { casebranches }
\end{codeEx}

       Detta kan bryta semantiken, tex om
\begin{codeEx}
f x z = case z of
    True  -> error -- tex en oändlig loop
    False -> x
\end{codeEx}
      Om vi här försöker optimera i brancherna då z är okänd kommer optimeringen
      att loopa, och det skulle programmet inte gjort om t ex. f alltid får False som
      argument.

\subsubsection{Inline-räknare}
     När man valt en utökning som bryter mot semantiken kan man inte alltid vara säker på
     att den terminerar, därför kan inline-räknaren användas för att garantera det det högst
     ser ett visst antal inlines.

     Det går att både begränsa det globala antalet inlines och inlines för olika funktioner.

\begin{codeEx}
     f z = case g z of
        A -> repeat 1
        B -> Nil
\end{codeEx}

    Här hade det varit otrevligt att försöka inlina repeat då den aldrig terminerar.


\end{document}


\documentclass[../Optimise]{subfiles}
\begin{document}

\section{Optimise ... with ...}
\label{sec:Optimise:With}

Det är finns specialfall där brott mot den lata STG-semantiken hade varit 
fördelaktigt.  Sådana fall är svåra att identifiera maskinellt, så därför 
kan användaren ge ytterligare argument till optimeringsfunktionen när 
de uppstår. Notationen är följande:

\begin{codeEx}
optimise f x with { options }
\end{codeEx}

Här är det möjligt att genom nyckelordet `\kw{with}' efter optimise ge olika direktiv
till optimeringen.

\paragraph{Case-grenar}
      Ibland finns det möjlighet att optimera i en \kw{case}-gren. Detta bryter semantiken
      då det kan leda till evaluering av uttryck som vanligtvis inte skulle ha beräknats,
      som exempelvis följande:
\begin{codeEx}
f x y z = case g z of
    { A -> case h x y of
        { R -> t1 z
        ; S -> t2 z
        }
    { B -> case h y x of
        { R -> t3 z
        ; S -> t4 z
        }
    };
\end{codeEx}

Vid evaluering av \miniCode{optimise(f 1 2 )} är \miniCode{z} okänd, men i
grenarna finns det \kw{case}-satser som bara beror på \miniCode{x} och \miniCode{y} 
och rätt gren kan alltså väljas. Det finns  stöd för att optimera i sådana här \kw{case}-grenar:

\begin{codeEx}
  optimise f x y with { casebranches }
\end{codeEx}

Detta kan ändra programs beteende, exempelvis här:

\begin{codeEx}
f x z = case z of
    { True  -> error
    ; False -> x
    };
\end{codeEx}

I kodexemplet kan \ic{error} till exempel vara en oändlig loop, eller något
som skapar ett fel när det körs. 
Om grenarna optimeras trots att \ic{z} är okänd kommer optimeringen
att loopa, och det skulle programmet inte ha gjort om t.ex. \ic{f} alltid får \ic{False} som
argument.

\paragraph{Infogningsräknare}
När en utökning som bryter mot semantiken är vald kan är det inte säkert
att programmet terminerar, på grund av oändliga infogningar. 
Därför kan en infogningsräknare användas för att garantera
att det det högst sker ett visst antal infogningar.

Det går både att begränsa det globala antalet infogningar samt antalet infogningar för specifika funktioner.

I ett kodavsnitt som använder \ic{repeat}-funktionen skulle det vara dumt att
försöka infoga den tills det inte går mer, eftersom funktionen är oändligt rekursiv,
men ibland kanske det skulle vara önskvärt att den iallafall är infogad ett par steg.
Det kan anges med hjälp av följande syntax som betyder att
$10$ infogningar av just den funktionen tillåts:

\begin{codeEx}
optimise ... with { inline repeat 10 }
\end{codeEx}

På ett liknande sätt skrivs en begränsning av antalet infogningar oavsett vilken funktion det gäller:

\begin{codeEx}
optimise ... with { inlinings 10 }
\end{codeEx}

\end{document}

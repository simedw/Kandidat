\documentclass[../Optimise]{subfiles}
\begin{document}

\subsection{Dödkods eliminering}

\NOTE{
\begin{align*}
\CBNPSI{\oLet xs}{\heap vh}A{lbs}v & \Rightarrow & \CBNPSI sHA{lbs}v\\
\end{align*}
}

När diverse optimeringar har körts och det är dags att bygga upp träded igen kommer
så kallad "död kod" att förekomma. Det vill säga, kod som inte längre behövs eftersom
den inte kan nås.

\begin{codeEx}
main = let t = getValue() in
            case True of
               True  -> 15
               False -> t
\end{codeEx}

Detta exemplet är trivialt att optimera då värdet i casen är känd. Vi kan därför
initisera den korrekta branchen.


\begin{codeEx}
main = let t = getValue() in 15
\end{codeEx}


Vi ser nu att t inte längre är bunded i uttrycket (15) och därför kan vi
 betrakta \miniCode{let t = getValue} som `död kod'. Död kod tar inte bara onödig plats
 vilket i sig är ett stort problem, kod ligger på flera pages i minnet och
 access tiden blir otroligt mycket högre, det tar även längre tid att traversera
 träded i interperten eller i andra optimeringar. Därför är det viktigt att
 eftersträva en så kort kod som möjligt.

\begin{codeEx}
main = 15
\end{codeEx}

Denna process kan mer formellt skrivas som

\begin{codeEx}
t not in freevars(e2)
++++++++++++++++++++++++++
let t = e1 in e2 => e2
\end{codeEx}

Detta är en kontroll som utförs varje gång vi bygger upp en let i Irr och Psi.
Dock är freevars ganska kostsam (bör stå om det här någon annanstans).


\end{document}

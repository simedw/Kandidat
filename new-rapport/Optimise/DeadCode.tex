\documentclass[../Optimise]{subfiles}
\begin{document}

\subsection{Dödkodseliminering}
\label{sec:DeadCode}

\NOTE{
\begin{align*}
\CBNPSI{\oLet xs}{\heap vh}A{lbs}v & \Rightarrow & \CBNPSI sHA{lbs}v\\
\end{align*}
}

När diverse optimeringar har körts och det är dags att bygga upp trädet igen kan
så kallad `död kod' att förekomma. Det vill säga, kod som inte längre 
behövs eftersom den inte kan nås.

\begin{codeEx}
main = let t = getValue in
            case True of
              { True  -> 15
              ; False -> t
              };
\end{codeEx}

Detta exemplet är trivialt att optimera då värdet i casegranskaren är känd. 
Den korrekta grenen kan alltså initieras.

\begin{codeEx}
main = let t = getValue in 15;
\end{codeEx}


Vi ser nu att t inte längre refereras till i uttrycket \miniCode{15} 
och därför kan \miniCode{let t = getValue} betraktas som `död kod'. 

Död kod tar onödig plats, de leder till att onödigt mycket kommer 
allokeras på heapen bara för att garbagecollectas. Main skulle
kunna optimeras till:

\begin{codeEx}
main = 15;
\end{codeEx}

Denna process kan mer formellt skrivas som

\begin{mathpar}
\inferrule
  {t\,not\,in\,freevars(e_2)}
  {\mathtt{let}\,t\,=\,e_1\,\mathtt{in}\,e_2 \Rightarrow e_2}
\;
\end{mathpar}

Detta är en kontroll som utförs varje gång vi bygger upp en \miniCode{let} 
i Irr och Psi. Dock är freevars en ganska kostsam operation så man kan fråga sig
om det är värt det.

Mer for
\NOTE{bör stå om det här någon annanstans, kanske i gc?}.


\end{document}

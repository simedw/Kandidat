n\documentclass[../Optimise]{subfiles}
\begin{document}

\subsection{Fallstudie}

\NOTE{
Turtoise: det är som ett gammal skämt, det tappar något

Achilles: vadå?

Turtoise: farten!!
}

Då det inte finns någon funktionalitet som beräknar var det skulle vara
bra att optimera så behöver användaren göra det själv, därför är det trevligt om
användaren får lite tips om var dessa annoteringar skall vara. Hur de kan påverka
resultatet.


Nu har det pratats mycket om optimise, men hur gör man nu när man ska använda
funktionaliten? I det här avsnittet kommer vi förklara var det är fördelaktigt
att skriva optimisenyckelordet, och var det får oeftersträvansvärt resultat.



Huvudtanken med optimise var att det är bra att köra den på en funktion
som senare kommer användas flera gånger. Typexempel på det här inom
funktionell programmering är \ic{map}-funktionen, här med powerexemplet:

\begin{codeEx}
main = map (power getInt) [1..20]
\end{codeEx}

Hur ska man resonera för att optimera exponentfunktionen här?
\ic{power getInt} ska köras på varje element i listan [1..20], så det 
är den funktionen som ska optimeras. Inte map, den står snarare för
distibutionen av att funktionen appliceras på varj element i listan.
\begin{codeEx}
-- rätt
main = map (optimise power getInt) [1..20]

-- fel
main = let f = optimise map (power getInt)
       in  f [1..20]
\end{codeEx}



Om man istället redan vet listan och vill få den att rullas ut och applicerar en
funktion på varje element i den kända listan. På detta sätt kommer man inte att 
att case:a på kända listan utan istället bygga upp med kända.
\begin{codeEx}
-- kvadrerar, kuberar och ökar inputlistan med 1
main = let g f = map f getIntList
       in  map g [square, cube, incr];
\end{codeEx}

Här måste man använda funktionen \ic{listSeq} som kommer att forcera alla element
i en lista för att sedan bygga upp den igen. Detta så att \kw{optimise} kommer
att få information om att den får forcera evaluering av resterande lista.. 
Minns att \kw{case}-satser forcerar ut evaluering, även under optimering.
\begin{codeEx}
listSeq li = case li of
    { Cons x xs = case x of
        { x' -> case listSeq xs of
            { xs' -> Cons x' xs'
            }
        }
    ; Nil -> Nil
    };
\end{codeEx}

Här ska du placera \kw{optimise} och \ic{listSeq}:
\begin{codeEx}
-- rätt
main = let g = optimise (\f . listSeq (map f getIntList))
       in  map g [square, cube, incr];

-- fel
main = let g = optimise (\f . map f getIntList)
       in  map g [square, cube, incr];
\end{codeEx}

%\subsection{Småexempel på mer man kan göra (med optimise)}
%
%Här följer några mindre exempel på vad resultatet av optimise blir:
%
%\begin{itemize}
%\item Funktionen \ic{foo} i följande exempel blir när den har applicerats på \ic{x}
%en funktion som tar ett argument, \ic{y}, och adderar eller subtraherar 1 från
%det beroende på värdet på \ic{x}.
%
%\begin{codeEx}
%foo x y = case x > 0 of
%    { True  -> y + 1
%    ; False -> y - 1
%    };
%\end{codeEx}
%Om vi vill köra den funktionen med samma \ic{x}-värde många gånger kan den optimeras
%med hjälp av \ic{optimise}, exempelvis såhär:
%\begin{codeEx}
%main = map (optimise (foo getInt)) getIntList;
%\end{codeEx}
%Om \ic{getInt} är 1 väljer optimeringsfunktionen grenen \ic{True} i \ic{case}-satsen:
%\begin{codeEx}
%optimise (foo 1) === \y . y + 1
%\end{codeEx}
%
%\item På liknande sätt som ovan kan funktioner som är rekursiva i ett argument
%vecklas ut. Följande funktion adderar två tal steg för steg och kan vecklas ut
%vid optimering:
%
%\begin{codeEx}
%add x y = case x > 0 of
%    { True  -> add (x - 1) (y + 1)
%    ; False -> y
%    };
%\end{codeEx}
%
%Vid optimerig av \ic{add 2} fås följande funktion, där \ic{case}-satserna och 
%rekursiva anrop har plockats bort eftersom värdet på \ic{x} är känt när
%optimeringen körs:
%
%\begin{codeEx}
%optimise (add 2) === \y . y + 1 + 1
%\end{codeEx}
%
%\item ... Kända funktioner och sånt.
%
%\end{itemize}

% radera inte mig :)
% inte mig heller :)
% ta de som é över :)

\end{document}
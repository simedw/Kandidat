\documentclass[../Appendix]{subfiles}
\begin{document}

\chapter{Print continuation}
\label{sec:printcont}
Utdata fås genom att \ic{main} evalueras fullt, till normalform. 
% Ytterligare två continuations behövs - \cont{Print} och \cont{PrintCont}. 
Uttrycksdatatypen utökas med en ny typ av värden, 
s-värden, som antingen är en evaluerad primitiv (betecknas $n$)
, till exempel ett heltal, eller en konstruktor (betecknas $K C [sval]$, där
$C$ är konstruktorns namn och sen listan av dess s-värden) .
Om något som ska skrivas ut har evaluerats till en (partiellt evaluerad)
funktion så skrivs \kw{FUN} istället ut. Detta är också ett s-värde.


%\begin{codeEx}  
%sval ::= n num
%       | K C [sval]
%       | <FUN>
%\end{codeEx}

%nya Cont:
%* Print
%* PrintCont C [sval] [atom]


Två nya continuations behövs, \cont{Print} och \cont{PrintCont C [sval] [atom]}.
\cont{Print} betyder att det som evalueras ska skrivas ut. \cont{PrintCont} består av ett
konstruktors namn \ic{C}, och först en lista på färdiga s-värden och en lista med
atomer som maskinen har kvar att evaluera.

Fem nya regler läggs till:

\[
x ; Print : s ; H[x \mapsto \CON C{a_1 \ldots a_n}]
\Rightarrow \begin{cases} 
K \; C \; [] ; S ; H , n = 0 \\
a_1 ; Print : PrintCon \; C \; [] \; [a_2 \ldots a_n] : S ; H , n > 0
\end{cases}
\]

Om x har evaluerats till en konstruktor \ic{C} finns två fall. Antingen är \ic{C}
en nullär konstruktor och i det fallet är vi klara och ger s-värdet bestående
av konstruktorn \ic{C}.
    Är det inte en nullär konstruktor behövs alla dess konstruerande atomer
evalueras, med start på $a_1$. Den läggs ut på maskinen med en \cont{Print}-continuation
och under denna en continuation som säger att de andra atomerna också ska
evalueras.
 

\[
x ; Print : S ; H \; \Rightarrow \; n \; x ; S ; H
\]

\ic{x} är alltså här ett värde, men ej ett s-värde.
Om \ic{x} är en primitiv datatyp, t.ex. en \ic{int} eller \ic{double}, så skapas detta s-värde
\ic{n x}.


\[
x ; Print : S ; H[x \mapsto FUN / PAP] \Rightarrow FUN ; S ; H
\]

Om \ic{x} pekar på en funktion skapas s-värdet $FUN$ för att visa att värdet har
evaluerats klart. 


\[
sv ; PrintCon \; C \; ps \; (n : ns) : S ; H \Rightarrow n ; Print : PrintCon \; C \; 
(ps ++ sv) \; ns : S ; H
\]

Om man har ett s-värde och en \cont{PrintCon}-continuation på stacken som har flera
atomer som man vill evaluera sätts det nuvarande värdet ihop med de redan färdigevaluerade
och man väljer nästa atom för att fortsätta att evaluera.

Till slut kommer man till en punkt då det inte finns fler atomer för denna konstruktor
att evaluera. Då kan konstruktorn byggas upp igen.

\[
sv ; PrintCon \; C \; ps \; [] : S ; H
\Rightarrow K \; C \; (ps ++ sv) ; S ; H
\]

Konstruktorn är helt evaluerad när den kommit till sista konstruerande atomen. Så
man får nu ett nytt s-värde som kan användas för att bygga upp nya konstruktorer.

\end{document}

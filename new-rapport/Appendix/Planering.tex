\documentclass[../Appendix]{subfiles}
\begin{document}

\chapter{Tidsplan}

Här följer projektets tidsplan, vilket är en obligatorisk del i kandidatarbetet,
men som även har fungerat väldigt väl: 

\paragraph{Milstolpe 2010-02-12 }
\begin{itemize}
\item Ha satt upp ett versionskontrolleringssystem för projektets kod.
\item Ha definierat en Haskelldatastruktur som representerar STG.
\item Ha definierat en Haskelldatastruktur som representerar resultatet
av en parsning.
\item Göra klart den här rapporten.
\end{itemize}

\paragraph{Milstolpe 2010-02-19}
\begin{itemize}
\item Standardbibliotek

\begin{itemize}
\item Listfunktioner: foldr, map, filter, length, replicate, repeat, iterate,
take, drop, head, tail
\item Funktionella funktioner: S, K, compose, flip, id
\item Tupelfunktioner: fst, snd, first, second
\item Booleska funktioner: \&\&, ||
\end{itemize}
\item En testsvit som testar alla ovan nämnda funktioner i standardbiblioteket
(och därmed också case- och let-uttryck).
\item En parser som kan parsa alla program i testsviten.
\item Ha en prettyprinter för STG-språket som kan skriva ut alla program
i testsviten.
\item Parsern och prettyprintern uppfyller $parse\circ pretty=id$.
\item Kunna tolka (köra) funktionsapplikation i STG.

\begin{itemize}
\item Undersöka och bestämma hur vi ska göra \textit{expression instantiation}%
\end{itemize}
\end{itemize}

\paragraph{Milstolpe 2010-03-05}
\begin{itemize}
\item Kunna avsockra funktionsapplikationer och case-satser från parsersyntaxträd
till STG, med scopeanalys om det behövs (beroende på hur vi ska göra
\textit{expression instantiation}):

\begin{codeEx}
fac x = case x == 0 of
	{ True -> 1
	; False -> x * fac (x - 1)
    };

main x = fac (head x);
\end{codeEx}
översätts till

\begin{codeEx}
fac = FUN (x -> case (==) x 0 of 
    { True  -> 1
    ; False -> let { p1 = THUNK ((-) x 1)
	               ; p2 = THUNK (fac p1)
	               } in (*) x p2 
    });

main = FUN (x -> fac x);
\end{codeEx}

\item Sammanställ tre idéer på icke-triviala exempelprogram som illustrerar
optimeringsfunktionens förträfflighet.
\item Kunna tolka (köra) let, case och de inbyggda primitiva funktionerna.
\end{itemize}

\paragraph{Milstolpe 2010-03-15}
\begin{itemize}
\item Köra programmens mainfunktion enligt basspecifikationen.
\item Göra optimeringarna nämnda i syfte under körningstid (caseeliminering
och konstantfolding).
\item Kunna mäta programmens snabbhet m.h.a. stegräknare.
\item Välja ut och skriva ett av de tre exempelprogrammen. Tolken ska kunna
köra programmet.
\item Förberedelser för halvtidsredovisningen som skall presenteras 16
mars.
\end{itemize}

\paragraph{Milstolpe 2010-03-19 (Fredag LV 1)}
\begin{itemize}
\item Ha gjort ett skelett för rapporten som innehåller rubriker och i grova
drag innehåll (punktlistor eller stolpar).
\item Kunna stänga av och på infogning med hjälp av en ny språkkonstruktion
(optimise with).


\begin{codeEx}
optimise power 4 with 
	{ inline     = power : if : Nil 
	; caseInCase = False
	}

opt f n = optimise f with 
	{ noInline  = take : Nil
	; maxInline = n
	}
\end{codeEx}

\item Analysera vilka optimeringar shapesexemplet skulle kunna dra nytta
av.
\end{itemize}

\paragraph{Milstolpe 2010-03-26 (Fredag LV 2)}
\begin{itemize}
\item Få optimeringen att enbart forcera thunkar som behövs. Detta gör så
att till exempel optimise av if-funktionen terminerar.
\item Kortare texter i rapporten där det är möjligt. I alla fall ska det
här innefatta tidigare arbeten, inledning/problemställning, beskrivning
av stg-maskinen, arbetsmetodik och den optimise-semantik som vid den
här tidpunkten är klar.
\end{itemize}

\paragraph{Milstolpe 2010-04-16 (Fredag LV 3)}
\begin{itemize}
\item Använda en anropsstack istället för (hjärndöd) substituering i tolken.

\begin{itemize}
\item Alla program ska fortfarande fungera med den här ändringen och de
ska vara snabbare.
\end{itemize}
\item Utkast till sammanfattning, resultat, diskussion för rapporten.
\end{itemize}

\paragraph{Milstolpe 2010-04-23 (Fredag LV 4)}
\begin{itemize}
\item $\alpha$-rapport. En sammanställd rapport som innehåller tillräckligt
mycket information för att en läsare ska kunna förstå den, vilket
möjliggör återkoppling. Detta betyder att våra poänger måste gå fram
men att den är så koncentrerad att vi snabbt kan få synpunkter på
den. $\alpha$-rapporten ska vara inlämningsbar men kanske inget vi
skulle vara stolta över.
\item Regexpexempel klart.
\end{itemize}

\paragraph{Milstolpe 2010-04-30 (Fredag LV 5)}
\begin{itemize}
\item En optimisefunktion som fungerar med callstacken, som har samma funktionalitet
som den förra call-by-name-optimeringen1.
\end{itemize}

\paragraph{Milstolpe 2010-05-07 (Fredag LV 6)}
\begin{itemize}
\item $\beta$-rapport. En enligt oss helt inlämningsbar (med stoltheten
och betyget i behåll) rapport som vi kan få synpunkter på.
\end{itemize}

\paragraph{Milstolpe 2010-05-17 (Måndag LV 8)}
\begin{itemize}
\item Helt klar rapport ändrad utifrån givna synpunkter och finslipad
på alla sätt och vis.
\end{itemize}

\paragraph{Milstolpe 2010-05-24 (Måndag {}``LV 9'')}
\begin{itemize}
\item Skriftlig opponering färdig. (Ska vara klar 26:e).
\item Demonstrationsbara exempel - shapes och regexp.
\item Förberett en slutpresentation med demo. (Slutpresentation 31:a)
\end{itemize}

\paragraph{Milstolpe 2010-05-28 (Fredag {}``LV 9'')}
\begin{itemize}
\item Allting färdigt, speciellt presentationsförberedelser och muntlig
opponeringsförberedelser. 
\item Melonfest!
\end{itemize}

\end{document}

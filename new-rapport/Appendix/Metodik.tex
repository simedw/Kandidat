\documentclass[../Appendix]{subfiles}
\begin{document}

\subsection{Metodik}

Detta arbete gjordes som ett 15hp-kandidatarbete i datavetenskap. Arbetet bestod
av flera viktiga moment som
    tidsplanering,
    handledningsmöten,
    rapportskrivning,
    utveckling av produkten och
    tidsrapportering i form av dagbok om vilket arbete som har skett. I det stora
var det en kurs i att organisera och genomföra ett större projekt i grupp.


Internt i gruppen har vi arbetat mycket tillsammans. Vi har träffats och
bestämt tillsammans vad som behöver göras, med hjälp av diskussion och med
tidplanen som skrevs dels i början av arbetet, och uppdaterades halvvägs. 
Några gånger gjordes små förändringar av tidsplanen när vi upptäckte
vad som vi inte skulle hinna med eller inte tänkt på med den tidigare planen.

\subsubsection{Verktyg}

    De primära delarna av projektet har varit kodning, testning och utveckling
av semantiska regler samt rapportskrivning. Till vår hjälp har vi använt
olika verktyg. Ett av dessa var att använda en projektor. Den har varit till
stor hjälp och låtit hela gruppen fokusera på samma sak samtidigt. Detta
kan vara svårt att göra på en liten datorskärm.

    För att underlätta samarbetet mellan varandra har vi använt programvarorna
\emph{git}\footnote{www.git-scm.com}, samt \emph{gobby}\footnote{gobby.0x539.de}.
Det förstnämnda, git, är ett versionshanteringssystem där vi lägger upp alla
filer i kodning och rapport. Där kan man också ha olika grenar, och det har 
vi använt vid stora förändringar, som när vi bytte optimeringssemantik eller
till anropsstack.

    Gobby är ett program som låter många användare modifiera samma filer
samtidigt. Man kan se det som ett \emph{multiplayer-notepad}. Det fungerar bra
tillsammans med projektor. Ibland när vi har haft mycket kod att implementera
har vi satt en person på att vara koordinator, som ger ut små
programmeringsuppgifter till de andra att skriva, exempelvis en funktion eller
en semantikregel, som sen följer arbetet på projektorn. Detta har varit 
effektivt.

    När vi skrivit rapport har vi också använt Gobby för att alla kunna
modifiera filerna samtidigt. Det kan vara praktiskt om exempelvis en person
korrekturläser, en skriver nytt material och en omstrukturerar. Rapporten 
är skriven i \LaTeX.

\subsubsection{Standardbibliotek och testsvit}

För att slippa skriva om alla enkla funktioner hela tiden gjorde vi ett litet standardbibliotek
som helt enkelt är en fil med en rad funktioner skriven i sockerspråket som läggs
till på slutet av den aktuella filen som ska köras.

Under tiden som vi jobbade med projektet skrev vi ofta små enkla
program för att testa ny funktionalitet. Vi samlade på oss en mängd sådana program av
olika karaktär och komplexitet och gjorde så att de kunde testas automatiskt, 
vilket var ett bra redskap för oss 
när vi gjorde ändringar i tolken eftersom vi då kunde se om något som tidigare
fungerat slutade fungera med en viss ändring. 

En testsvit är också en bra morot när man gör stora ändringar som kanske gör så att
alla testprogram slutar fungera, för när man så sakteliga får saker att fungera
igen kommer det att reflekteras i antalet fungerande testprogram i testsviten. 
Man ser då konkreta resultat av sitt arbete.



% Varje medlem i projektet ska skriva sin 
% lilla historia om hur projektet har varit.
% .........eller??!!






\end{document}

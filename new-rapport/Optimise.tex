
\documentclass[Rapport]{subfiles}
\begin{document}

\section{Optimise}

% svårt att låta bli att alla dessa avsnitt börjar med 
% 'i det här avsnittet [ ska vi undersöka / beskrivs / förklaras ... ]'
% plepp, plepp

Nu är sockerpråket förklarat (sektion 2), och corespråket (sektion 3),
och nu kan vi undersöka själva kärnan i arbetet, nämligen optimeringen.
I introduktionen i detta avsnitt kan du läsa om vad optimering och 
optimering under körtid innebär. För att optimera under körtid har 
olika semantiker tagits fram och undersökts.
    
Den första av dem, som kommer refereras till som cbv-semantiken, 
kan du läsa om i avsnitt 4.2. Den hade vissa nackdelar, så i 
efterföljande avsitt kommer den uppgradera versionen, cbn-semantiken.

Denna kunde också tjäna på andra optimeringar, som tex
dödkodseliminering (sektion 4.5), 
välkänd caselag (sektion 4.6) och 
afterburner (sektion 4.7).

Hur optimeringen fungerar tillsammans med anropssatcken
kan du läsa i det sista avsnittet 4.8.

\subfile{Optimise/Intro}

\NOTE{
Den här texten är bra - var ska den in? Något för övergångstexten?
Eller i introduktionen? /Dan

Vi har använt ett par olika semantiker för att beskriva optimeringsskedet
i vårt språk, vi ska nu förklara var och ett av dessa i kronologisk
ordning.


Detta optimeringspass arbetar sig igenom trädet på ett sätt inte
helt olikt STG-maskinen, dock kan vi bland annat gå in och evaluera
funktion utan att känna till alla argument, s.k. evaluera under lambdat.
För att inte duplicera för mycket av STGs funktionalitet så anropas
optimise STG-maskinen för diverse uppgifter.
}

\subfile{Optimise/CBV}

\subfile{Optimise/With}

\subfile{Optimise/CBN}
\subfile{Optimise/DeadCode}
\subfile{Optimise/ValkandCaseLag}
\subfile{Optimise/AfterBurner}

\end{document}

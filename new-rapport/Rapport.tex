\documentclass[leqno]{article}

\usepackage{subfiles}
\usepackage{fegr}
\usepackage[T1]{fontenc}
\usepackage[utf8]{inputenc}
\usepackage[swedish]{babel}

\usepackage{stmaryrd}

%Stg.tex
\usepackage{listings}
\usepackage{fancybox}
\usepackage{calc}
\usepackage{amsmath}
\usepackage{xargs}[2008/03/08]
% End Stg.tex
% Opt1
\usepackage{amssymb}
%End Opt1

\begin{document}

%det är ju alltid viktigt att poängtera att jag går på GU :)
% jätteviktigt! eller vill du att det ska stå FEL!?
% tänker att... ääh
% du är iofs på chalmersområdet väldigt mycket!
% lika mycket gu-område faktiskt!
% men jag vill ha ny framsidan på en egen sida?
% ja, det är tanken. det var den när vi hade document class report ahoy

\title{Optimering under körningstid}
\date{}
\author{Simon  Edwardsson
   \and Olle   Fredriksson
   \and Daniel Gustafsson
   \and Dan    Rosén
   }
   %det här är ju en jättekonstig framsida :)
   % DU är en jättekonstig framsida :)
   
\maketitle

\subfile{Abstract}

\newpage

\tableofcontents

\newpage

\subfile{Inledning}

\subfile{Socker}

\subfile{Core}

\subfile{Optimise}

\subfile{Resultat}

\subfile{Diskussion}

\subfile{RelateradeArbeten}

\subfile{FramtidaArbeten}

\appendix

\newpage

\subfile{Appendix}

\end{document}

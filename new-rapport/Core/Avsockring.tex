\documentclass[../Core]{subfiles}
\begin{document}
\subsection{Avsockring till STG}
\label{sec:Diabetes}

\overviewSugar

STG är ett ganska primitivt språk där man t.ex. bara kan anropa funktioner
och skapa konstruktorer med argument som är atomer. Alla objekt måste även
explicit annoteras med information om huruvida de är thunkar eller konstruktorer. För att underlätta
att skriva mycket kod skapade vi ett sockerspråk, och ett
pass som går igenom koden och avsockrar den till STG. Här är ett exempel:

\begin{codeEx}
isPrimer n t = case t * t >= n of
    { True -> True
    ; False -> if (n % t == 0) False (isPrimer n (t + 1))
    };
\end{codeEx}

Den här koden kommer från testsviten och är del i ett program som testar
primalitet. Här används sockerspråket flitigt för att göra koden läsbar
och hanterlig. I exemplet är \miniCode{isPrimer} ett funktionsobjekt som i 
STG skulle skrivas mer i stil med en lambdafunktion:

\begin{codeEx}          
isPrimer = FUN (n t -> ...)
\end{codeEx}


I STG tar funktioner endast atomer som argument. Detta gör att 
jämförelsen \miniCode{t * t >= n} först behöver binda kvadreringen
till en temporär variabel:

\begin{codeEx}
let temp = THUNK (t * t)
in  (>=) temp n
\end{codeEx}

Liknande för \ic{if}-funktionen: \ic{if}s argument \miniCode{(n \% t == 0)} och 
\miniCode{isPrimer n (t + 1)} är inte atomer så de behövs bindas och vidare
är \miniCode{n \% t} och \miniCode{t + 1} inte atomer i  
funktionsanropen till \miniCode{==} och \miniCode{isPrimer}.


Allt med inledande versal i sockerspråket tolkas som konstruktorer,
precis som i Haskell. Så \miniCode{True} i koden ovan behöver allokeras med
en \kw{let}. Den första grenen blir i så fall
  \miniCode{True -> let temp = CON (True) in temp}.
Faktum är att alla nullära konstruktorer som \miniCode{True}, \miniCode{False}, \miniCode{Nil}, finns 
allokerade på toppnivå i tolken och instantieras endast en gång. 
Men i alla andra fall behöver de \kw{let}-bindas.

Detta gäller även heltal och andra primitiver. $0$ avsockras till
    \miniCode{let temp = CON (I\# 0) in  temp}
. Notera att alla temporära variabler som skapas är unika.
            
    
För att göra det lättare att operera på listor och göra booleska
jämförelser finns infixoperatorerna \ic{\&\&} och \ic{||} som alias för and respektive or
samt \ic{++} och \ic{:} för \ic{append} och \ic{Cons}. Under avsockringen byts dessa
infixvarianter ut mot motsvarande vanliga funktioner. 


I Haskell används också \ic{\$} flitigt som funktionsapplikation med låg
högerfixitet, exempelvis i denna kod som kör quicksort på en
lista sorterad i fallande ordning:

\begin{codeEx}
main = qsort $ reverse $ take 3 $ iterate (\x.x + 1) 0;
\end{codeEx}

Detta avsockras till motsvarande funktionsanrop:

\begin{codeEx}
main = qsort (reverse (take 3 (iterate (\x.x + 1) 0)));
\end{codeEx}

En annan metod är att göra \ic{\$} ett alias för en funktion \ic{apply} som
helt enkelt utför ett funktionsanrop:

\begin{codeEx}
main = apply (qsort 
     ( apply reverse 
     ( apply (take 3) (iterate (\x.x + 1) 0)));

apply f x = f x;
\end{codeEx}

Detta ger så stor extrakostnad för något så simpelt som funktionsapplikation
att det är bättre att göra det till funktionsanrop direkt.
        


%    \NOTE{Detta kanske borde vara i optimise-avsnittet} 
%    \subsubsection{Avsockring av optimise with}
%    \begin{codeEx}
%    let f = optimise foo x y with { caseBranches, inlinings = n * n } 
%    in map f zs
%    \end{codeEx}
%    avsockras till
%    \begin{codeEx} 
%    let { temp1 = THUNK ( n * n )
%        ; temp2 = OPT ( foo x y ) with { caseBranches, inlinings = temp1 }
%        ; f = THUNK ( case temp1 of _ -> temp2 }
%        }
%    in map f zs
%    \end{codeEx}
%        I sockerspråket är optimise ett nyckelord som ser ut som en 
%        funktionsapplikation, men avsockras till ett OPT-objekt. Dessutom
%        forceras alla dess inställningar med hjälp av \kw{case}-satser.%    


\subsection{Lambdalyftare}
\label{sec:LamLift}

I tolkandet av koden underlättar det om man inte har några \kw{let}-bundna funktioner.
Dessa behöver lyftas ut till toppnivå. I sockret kan dessa uppstå antingen
av att man skriver lambdafunktioner eller \kw{let}-bundna funktioner. Ett exempel
från quicksort:

\begin{codeEx}
  Cons x xs ->
      let { lesser y = y < x
          ; listLess = filter lesser xs
          ; listMore = filter (\y . y >= x) xs
          }
      in  qsort listLess ++ x : qsort ListMore
\end{codeEx}

Först översätts \kw{let}-satsen med hjälp av avsockraren

\begin{codeEx}
  .. let { lesser = FUN (y -> let t = THUNK (y < x) in t)
         ; listLess = THUNK (filter lesser xs)
         ; lambda0 = FUN (y -> let t = THUNK (y >= x) in t)
         ; listMore = THUNK (filter lambda 0 xs)
         } ..
\end{codeEx}
      
När dessa funktioner \ic{lesser} och \ic{lambda0} lambdalyfts, skapas nya funktioner
på toppnivå. En sak att ha i åtanke då är att de fria variablerna i funktionskroppen 
kan komma från ursprungsfunktionen. Dessa måste skickas med
som extra parametrar. I det här fallet rör det sig om variabeln \ic{x} i båda fallen.
Dessa parametrar måste också skickas med när de anropas.
%s/kan/nödvändigtvis
\begin{codeEx}
lesserLifted  = FUN(x y -> let t = THUNK (y <  x) in t)
lambda0Lifted = FUN(x y -> let t = THUNK (y >= x) in t)

    .. let { lesser   = THUNK (lesserLifted   x)
           ; listLess = THUNK (filter lesser  xs)
           ; lambda0  = THUNK (lambda0Lifted  x)
           ; listMore = THUNK (filter lambda0 xs)
           } ..
\end{codeEx}

Notera att \miniCode{listLess} och \miniCode{listMore} 
inte behöver modifieras eftersom de lyftna funktionerna partialevalueras 
med samma namn. \miniCode{lesser} är här funktionen \miniCode{lesserLifted}
partiellt applicerad på \miniCode{x}.


\end{document}

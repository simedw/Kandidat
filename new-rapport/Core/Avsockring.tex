\documentclass[../Core]{subfiles}
\begin{document}
\subsection{Avsockring till STG}
STG är ett ganska primitivt språk med tex att man kan bara anropa funktioner
och skapa konstruktorer med argument från atomer, samt att alla objekt måste
explicit annoteras om de är thunkar eller konstruktörer. För att underlätta
att skriva kod skapade vi ett sockerspråk för att underlätta detta, och ett
pass över koden som avsockrar.

\begin{codeEx}
isPrimer n t = case t * t >= n of
    { True -> True
    ; False -> if (n % t == 0) False (isPrimer n (t + 1))
    };
\end{codeEx}

Den här koden kommer från testsviten och är del i ett program som testar
primalitet. Här används sockerspråket flitigt för att göra koden läsbar
och hanterlig. I exemplet ovan är \miniCode{isPrimer} är ett funktionsobjekt och i 
STG och skrivs det i en stil mer lik en lambdafunktion:

\begin{codeEx}          
isPrimer = FUN (n t -> ...)
\end{codeEx}


I STG tar funktioner endast atomer som argument. Detta gör att 
i jämförelsen \miniCode{t * t >= n} behövs först kvadreringen sparas bindas
till en temporär variabel:

\begin{codeEx}
let temp = THUNK (t * t)
in  (>=) temp n
\end{codeEx}

Liknande för if-satsen: ifs argument \miniCode{(n \% t == 0)} och 
\miniCode{isPrimer n (t + 1)} är inte atomer så de behövs bindas och i
sin tur är \miniCode{n \% t} och \miniCode{t + 1} inte atomer till dessa 
funktionsanrop till \miniCode{==} och \miniCode{isPrimer}.


Allt med inledande versal i sockerspråket tolkas som konstruktorer,
precis som i Haskell. Så \miniCode{True} ovan i koden behöver allokeras med
en let. Den första grenen blir i så fall
  \miniCode{True -> let temp = CON (True) in temp}.
Faktum är att de nullära konstruktorerna \miniCode{True}, \miniCode{False}, \miniCode{Nil}, finns 
allokerade på toppnivå i tolken och instantieras endast en gång. 
Men i alla andra fall behövs de let-bindas.

Detta gäller även heltal och andra primitiver. $0$ avsockras till
    \miniCode{let temp = CON (I\# 0) in  temp}
Notera att alla temporära variabler som skapas är unika.
            
    
För att göra det lättare att operera på listor och göra booleska
jämförelser finns infixoperatorerna \&\&, || som alias för and och or
respektive, och ++ och : för append och cons. Funnes någon av dessa
infixvarianter byts de ut under avsockringen. 


I haskell används också \$ flitigt som funktionsapplikation med låg
högerfixitet, exempelvis i denna kod som testar quicksort med en
lista sorterad i fallande ordning:

\begin{codeEx}
main = qsort $ reverse $ take 3 $ iterate (\x.x + 1) 0;
\end{codeEx}

Detta avsockras till motsvarande funktionsanrop:

\begin{codeEx}
main = qsort (reverse (take 3 (iterate (\x.x + 1) 0)));
\end{codeEx}
      En annan metod är att göra \$ ett alias för apply:
\begin{codeEx}
main = apply (qsort (apply reverse (apply take3 (iterate (\x.x + 1) 0)));

apply f x = f x;
\end{codeEx}

Detta ger så stor overhead för något så simpelt som funktionsapplikation
att det är rimligare att göra det till funktionsanrop direkt.
        



\NOTE{Detta kanske borde vara i optimise-avsnittet} 
Avsockring av optimise with

\begin{codeEx}
let f = optimise foo x y with { caseBranches, inlinings = n * n } 
in map f zs
\end{codeEx}

avsockras till

\begin{codeEx} 
let { temp1 = THUNK ( n * n )
    ; temp2 = OPT ( foo x y ) with { caseBranches, inlinings = temp1 }
    ; f = THUNK ( case temp1 of _ -> temp2 }
    }
in map f zs
\end{codeEx}

    I sockerspråket är optimise ett nyckelord som ser ut som en 
    funktionsapplikation, men avsockras till ett OPT-objekt. Dessutom så 
    forceras alla dess inställningar med hjälp av en casesats.


\subsubsection{Lambdalyftare}

I tolkandet av koden underlättar det om man inte har några let-bundna funktioner.
Dessa behöver lyftas ut tilkl toppnivå. I sockret kan dessa uppstå antingen
av att man skriver lambdafunktioner eller let-bundna funktioner. Ett exempel
från quicksort:

\begin{codeEx}
  Cons x xs ->
      let { lesser y = y < x
          ; listLess = filter lesser xs
          ; listMore = filter (\y . y >= x) xs
          }
      in  qsort listLess ++ x : qsort ListMore
\end{codeEx}

Först översätts let-satsen med hjälp av avsockraren

\begin{codeEx}
  .. let { lesser = FUN (y -> let t = THUNK (y < x) in t)
         ; listLess = THUNK (filter lesser xs)
         ; lambda0 = FUN (y -> let t = THUNK (y >= x) in t)
         ; listMore = THUNK (filter lambda 0 xs)
         } ..
\end{codeEx}
      
När dessa funktioner, lesser och lambda0, lambdalyfts, skapas nya funktioner
på toppnivå. En sak att ha i åtanke då är att de fria variablerna i funktions-
kroppen nödvändigtvis kommer från ursprungsfunktionen. Dessa måste skickas med
som extra parametrar. I det här fallet rör det sig om variabeln x i båda fall.
Dessa parametrar måste också skickas med när de anropas.

\begin{codeEx}
lesserLifted = FUN(x y -> let t = THUNK (y < x) in t)
lambda0Lifted = FUN(x y -> let t = THUNK (y >= x) in t)

    .. let { lesser = THUNK (lesserLifted x)
           ; listLess = THUNK (filter lesser xs)
           ; lambda0 = THUNK (lambda0Lifted x)
           ; listMore = THUNK (filter lambda0 xs)
           } ..
\end{codeEx}

En sak att poängtera är att \miniCode{listLess} och \miniCode{listMore} 
inte behövs modifieras för att de lyftna funktionerna partialevalueras 
med samma namn. Här \miniCode{lesser} som blir \miniCode{lesserLifted}
partiellt applicerat på \miniCode{x}.


\end{document}

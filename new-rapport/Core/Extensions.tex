\documentclass[../Core]{subfiles}
\begin{document}
\subsection{Utökningar}

\subsubsection{Boxing}

För att få en uniform representation av användardefinierade datatyper, skapade
med en konstruktor, och primitiva datatyper som \ic{Int}, \ic{Double} och \ic{Char}, ges även 
de sistnämnda konstruktorer. \miniCode{I\#} för \ic{Int}, \miniCode{D\#} för \ic{Double} och 
\miniCode{C\#} för \ic{Char}. De har invarianten, till skillnad från icke-primitiva datatyper, 
att deras argument inte pekar på en thunk utan alltid måste vara fullt evaluerade. 
Detta kallas för att `boxa' primitiven.

    För addition skapas en funktion som avboxar primitiven,
forcerar och utför primitiv addition och sedan boxar igen:

\begin{codeEx}
x + y = case x of 
    { I# x' -> case y of
        { I# y' -> case x' +# y' of
            { r -> let r' = CON (I# r) in  r'}}};
\end{codeEx}

Funktionen \ic{+} är den operator för addition som användaren av vårt språk får använda medan $ +\# $ är den
primitiva additionen som inte exporteras till användaren. Notera att $ x' +\# y' $
forceras med hjälp av en \kw{case}-sats för att det resulterande talet ska vara fullt
evaluerat före det boxas igen.

\subsubsection{IO}
\NOTE{Oh Hi, vi borde förklara skillnad mellan NF och WHNF}
Tolken har en primitiv form av IO: den har konstant indata vid körning och
skriver ut det som \ic{main}-funktionen returnerar. Dessutom
forceras hela värdet om det är en konstruktor till normalform. Med det menas
att värdet evalueras helt och hållet och inte bara till vilken konstruktor det är
som man gör i WHNF.

Funktionerna getInt, getIntList, getDouble, getDoubleList och getString finns
till hands under körning om användaren har angett någon eller några av dessa
i anropet till tolken. Dessa är konstanta under körningstid och gör det
lättare att testa programmen genom att ändra deras indata utan att modifiera
källkoden.

\subsubsection{Print continuation}
\NOTE{Finns lite arbete och göra på dessa}
Utdata fås genom att \ic{main} evalueras fullt. % Ytterligare två continuations behövs - \cont{Print} och \cont{PrintCont}. 
Uttrycksdatatypen utökas med en ny typ av värden, 
s-värden, som antingen är en evaluerad primitiv (betecknas $n$)
, t.ex. ett heltal, eller en konstruktor (betecknas $K C [sval]$, där
$C$ är konstruktorns namn och sen listan av dess s-värden) .
Om något som ska skrivas ut har evaluerats till en (partiellt evaluerad)
funktion så skrivs \kw{FUN} istället ut. Detta är också ett s-värde.


%\begin{codeEx}  
%sval ::= n num
%       | K C [sval]
%       | <FUN>
%\end{codeEx}

%nya Cont:
%* Print
%* PrintCont C [sval] [atom]


Två nya continuations behövs, \cont{Print} och \cont{PrintCont C [sval] [atom]}.
\cont{Print} betyder att det som evalueras ska skrivas ut. \cont{PrintCont} består av ett
konstruktorsnamn C, och först en lista på färdiga s-värden och en lista med
atomer som maskinen har kvar att evaluera.

Fem nya regler läggs till:
\[
x ; Print : s ; H[x \mapsto \CON C{a_1 \ldots a_n}]
\Rightarrow \begin{cases} 
K \; C \; [] ; s ; H , n = 0 \\
a_1 ; Print : PrintCon \; C \; [] \; [a_2 \ldots a_n] : s ; H , n > 0
\end{cases}
\]

Om x har evaluerats till en konstruktor C så finns två fall. Antingen att C är
en nullär konstruktor och i det fallet är vi klara och ger s-värdet bestående
av konstruktorn C.
    är det inte en nullär konstruktor behövs alla dess konstruerande atomer
evalueras, med start på $a_1$. Den läggs ut på maskinen med en \cont{Print}-continuation
och under denna en contiunation som säger att de andra atomerna också ska
evalueras.
 

\[
x ; Print : s ; H \;
\Rightarrow \; n \; x ; s ; H
\]

x är alltså här ett värde, men ej ett s-värde.
Om x är en primitiv datatyp, tex en int eller double, så skapa detta s-värde
\ic{n x}.


\[
x ; Print : s ; H[x \mapsto FUN / PAP]
\Rightarrow FUN ; s ; H
\]

Om x pekar på en funktion skapa s-värdet $FUN$ för att visa att att värdet har
evaluerats klart. vi väljer att bara visa $FUN$ istället för att 


\[
sv ; PrintCon \; C \; ps \; (n : ns) : s ; H
\Rightarrow n ; Print : PrintCon \; C \; (ps ++ sv) \; ns : s ; H
\]

Om man har ett s-värde och en \cont{PrintCon}-continuation på stacken som har flera
atomer som man vill evaluera så sätts nuvarande värdet ihop med de redan färdigevaluerade
och man väljer nästa atom för att fortsätta att evaluera.

Till slut kommer man komma till att det inte finns fler atomer för denna konstruktor
att evaluera, då kan Bygga upp konstruktorn igen.

\[
sv ; PrintCon \; C \; ps \; [] : s ; H
\Rightarrow K \; C \; (ps ++ sv) ; s ; H
\]

Konstruktorn är helt evaluerad när den kommit till sista konstruerande atomen. Så
man får nu ett nytt s-värde som kan användas för att bygga upp nya kontstruktorer.



\end{document}

\documentclass[../Core]{subfiles}
\begin{document}

% \subsection{Utökningar}

\paragraph{Normalform}
\label{sec:whnf}
\label{sec:nf}

När tolken kör programmet evalueras main till normalform.
Att evaluera ett uttryck tills konstruktorn är känd, men dess argument kan 
fortfarande vara thunkar, kallas vek normalform 
(engelska weak head normal form, förkortas whnf). Normalform, däremot, så
forceras alla konstruktorns argument att också evalueras till normalform.
Alltså till ett värde, dvs konstruktor eller funktion. Reglerna för dessa
finns i appendix \ref{sec:printcont}

\paragraph{Boxing}
\label{sec:boxing}

För att få en uniform representation av användardefinierade datatyper, skapade
med en konstruktor, och primitiva datatyper som \ic{Int}, \ic{Double} och \ic{Char}, ges även 
de sistnämnda konstruktorer. \miniCode{I\#} för \ic{Int}, \miniCode{D\#} för \ic{Double} och 
\miniCode{C\#} för \ic{Char}. De har invarianten, till skillnad från icke-primitiva datatyper, 
att deras argument inte pekar på en thunk utan alltid måste vara fullt evaluerade  \cite{santos}. 
Detta kallas för att `boxa' primitiven.

    För addition skapas en funktion som avboxar primitiven,
forcerar och utför primitiv addition och sedan boxar igen:

\begin{codeEx}
x + y = case x of 
    { I# x' -> case y of
        { I# y' -> case x' +# y' of
            { r -> let r' = CON (I# r) 
                   in  r'
    }   }   };
\end{codeEx}

Funktionen \ic{+} är den operator för addition som användaren av språket
får använda medan \ic{+\#} är den primitiva additionen som inte exporteras till 
användaren. Notera att \ic{x' +\# y'}
forceras med hjälp av en \kw{case}-sats för att det resulterande talet ska vara fullt
evaluerat före det boxas igen.

Då språket inte har ett typsystem skapades en speciell funktion för primitiver
på \kw{Double}. Precis som i OCaml\footnote{http://ocaml.inria.fr} har dessa
givits namn med \ic{.} som suffix, exempelvis \ic{+.}, \ic{*.}, \ic{<.}.
Liknande namngivningsschema finns för \kw{Char}, men med \ic{:} som suffix,
exempelvis görs likhet på \kw{Char} görs med \ic{==:}.

\paragraph{IO}
Tolken har en primitiv form av IO: den har konstant indata vid körning och
skriver ut det som \ic{main}-funktionen returnerar. Funktionerna 
\ic{getInt}, 
\ic{getIntList},
\ic{getDouble},
\ic{getDoubleList} och 
\ic{getString} 
finns till hands under körning om användaren har angett någon eller några av
dessa i anropet till tolken. Dessa är konstanta under körningstid och gör det
lättare att testa programmen genom att ändra deras indata utan att modifiera
källkoden.


\end{document}

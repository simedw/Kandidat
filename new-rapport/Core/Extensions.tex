\documentclass[../Core]{subfiles}
\begin{document}

% \subsection{Utökningar}

\paragraph{Boxing}
\label{sec:boxing}

För att få en uniform representation av användardefinierade datatyper, skapade
med en konstruktor, och primitiva datatyper som \ic{Int}, \ic{Double} och \ic{Char}, ges även 
de sistnämnda konstruktorer. \miniCode{I\#} för \ic{Int}, \miniCode{D\#} för \ic{Double} och 
\miniCode{C\#} för \ic{Char}. De har invarianten, till skillnad från icke-primitiva datatyper, 
att deras argument inte pekar på en thunk utan alltid måste vara fullt evaluerade  \cite{santos}. 
Detta kallas för att `boxa' primitiven.

    För addition skapas en funktion som avboxar primitiven,
forcerar och utför primitiv addition och sedan boxar igen:

\begin{codeEx}
x + y = case x of 
    { I# x' -> case y of
        { I# y' -> case x' +# y' of
            { r -> let r' = CON (I# r) 
                   in  r'
    }   }   };
\end{codeEx}

Funktionen \ic{+} är den operator för addition som användaren av språket
får använda medan \ic{+\#} är den primitiva additionen som inte exporteras till 
användaren. Notera att \ic{x' +\# y'}
forceras med hjälp av en \kw{case}-sats för att det resulterande talet ska vara fullt
evaluerat före det boxas igen.

Då vårt språk inte har ett typsystem skapades separata funktioner för primitiva operationer
på \kw{Double}. Inpirerat av OCaml\footnote{http://ocaml.inria.fr} har dessa
givits namn med \ic{.} som suffix, exempelvis \ic{(+.)}, \ic{(*.)}, \ic{(<.)} och så vidare.
Liknande namngivningsschema finns för \kw{Char}, men med \ic{:} som suffix,
där exempelvis likhet görs med \ic{(==:)} på \kw{Char}.

\paragraph{Normalform}
\label{sec:whnf}
\label{sec:nf}

När tolken kör programmet evalueras \ic{main}-funktionen till normalform.
Att evaluera ett uttryck tills konstruktorn är känd, där dess argument 
fortfarande kan vara thunkar, kallas som nämnt i det föregående kapitlet \emph{vek normalform}. 
För \emph{Normalform} däremot, betyder att även
konstruktorns alla argument forceras så att de också är värden, dvs konstruktorer eller funktioner.
Reglerna för att evaluera till normalform
finns i appendix \ref{sec:printcont}.


\paragraph{IO}
Tolken har en primitiv form av IO: den har konstant indata vid körning och
skriver ut det som \ic{main}-funktionen returnerar på normalform. Funktionerna 
\ic{getInt}, 
\ic{getIntList},
\ic{getDouble},
\ic{getDoubleList} och 
\ic{getString} 
finns till hands under körning om användaren har angett någon eller några av
dessa i anropet till tolken. Dessa är konstanta under körningstid och gör det
lättare att testa programmen genom att ändra deras indata utan att modifiera
källkoden.


\end{document}

\documentclass[../Core]{subfiles}
\begin{document}
\subsection{Våra utökningar}

\subsubsection{Boxing}

För att få en uniform representation av användardefinierade datatyper, skapade
med en konstruktor, och primitiva datatyper som Int, Double och Char, ges även 
de sistanämnda konstruktorer. I\# för Int, D\# för Double, C\# för Char. De har
invarianten, till skillnad från egna datatyper, att de inte pekar på en thunk
utan är fullt evaluerade. Detta kallas att boxa primitiven.
    För att utföra addition skapas den här funktionen som avboxar primitiven,
forcerar och utför primitiv addition och sen boxar igen:

\begin{codeEx}
x + y = case x of 
    { I# x' -> case y of
        { I# y' -> case x' +# y' of
            { r -> let r' = CON (I# r) in  r'}}};
\end{codeEx}

Så + är additionen som användaren av vårt språk får använda och $ +\# $ är den
primitiva additionen som inte exporteras för användaren. Notera att $ x' +\# y' $
forceras med hjälp av case för att den ska vara fullt evaluerad.

\subsubsection{IO}
Tolken har en primitiv form av IO: den har konstant indata vid körning och
skriver ut det main returnerar, som behöver vara en konstruktor. Dessutom
forceras hela konstruktorn till reducerad normalform.

Funktionerna getInt, getIntList, getDouble, getDoubleList och getString finns
till hands under körning om användaren har angett någon eller några av dessa
i anropet till tolken. Dessa är konstanta under körningstid och gör det
lättare att testa programmen genom att ändra deras indata utan att modifiera
källkoden.

\subsubsection{Print continuation}
\NOTE{Finns lite arbete och göra på dessa}
Utdata fås genom att main evalueras fullt. Två till continuations behövs - 
Print och PrintCont. Uttrycksdatatypen utökas med en ny typ av värden, sk
s-värden, som antingen är en evaluerad primitiv, tex en int, eller en
konstruktor. Om någon som ska skrivas ut evalueras till en (partiellt evaluerad)
funktion så skrivs <FUN> istället ut. Detta är också ett s-värde.

\begin{codeEx}  
sval ::= n num
       | K C [sval]
       | <FUN>
\end{codeEx}

nya Cont:

* Print
* PrintCont C [sval] [atom]

Print betyder att det som evalueras ska skrivas ut. PrintCont består av ett
konstruktorsnamn C, och först en lista på färdiga s-värden och en lista med
atomer som maskinen har kvar att evaluera.

nya regler:

(1)
\[
x ; Print : s ; H[x \mapsto \CON C{a_1 \ldots a_n}]
=> \begin{cases} 
K C [] ; s ; H , n = 0 \\
a_1 ; Print : PrintCon C [] [a_2 \ldots a_n] : s ; H , n > 0
\end{cases}
\]

Om x har evaluerats till en konstruktor C så finns två fall. Antingen att C är
en nullär konstruktor och i det fallet är vi klara och ger s-värdet bestående
av konstruktorn C.
    är det inte en nullär konstruktor behövs alla dess konstruerande atomer
evalueras, med start på a1. Den läggs ut på maskinen med en print-continuation
och under denna en contiunation som säger att de andra atomerna också ska
evalueras.
 
(2)
x ; Print : s ; H , x value other than sval
=> n x ; s ; H

Om x är en primitiv datatyp, tex en int eller double, så skapa detta s-värde.

(3)
x ; Print : s ; H[x -> FUN / PAP]
=> <FUN> ; s ; H

Om x pekar på en funktion skapa s-värdet <FUN>.

(4)
sv ; PrintCon C ps (n : ns) : s ; H
=> n ; Print : PrintCon C (ps ++ sv) ns : s ; H

Ett s-värde är fullt evaluerat och då kan nästa atom i konstruktorn evalueras.  

(5)
sv ; PrintCon C ps [] : s ; H
=> K C (ps ++ sv) ; s ; H

Konstruktorn är helt evaluerad när den kommit till sista konstruerande atomen.


\end{document}

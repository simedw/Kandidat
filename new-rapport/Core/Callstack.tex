\documentclass[../Core]{subfiles}
\begin{document}

\section{Anropsstack}
\label{sec:CallStack}
% Termen substitution används både för den abstrakta att byta ut, 
% samt för den hjärndöda substituionen
% vi behöver nog ett namn för hjärndöd substitution: Cerebellum disconectus?}
% \subsubsection{Aktiveringsposter}

Till en början använde vår tolk sig av naiv substitution, vilket innebär att man
traverserar syntaxträdet och byter ut variabler mot deras värde, vid exempelvis
funktionsapplikation.
Eftersom substitution är långsamt, behövs ett annat sätt att lösa problemet
på. Ett sätt att göra det på är att använda något som kallas för anropsstack, som
är vad många kompilatorer använder. % referens på det? :)

Varje funktion får vid anrop en egen aktiveringspost, som innehåller dess argument.
I funktionen blir alla argument något som kallas för lokala variabler, vilka kan representeras
som ett index in i funktionens aktiveringspost. Även variabler som binds i
\kw{let}-uttryck läggs i aktiveringsposten istället för på heapen. 
Eftersom anrop nästlas som vi kommer att se senare i avsnittet
behövs en stack av aktiveringsposter, som man kan pusha och poppa poster ifrån.
Denna stack kallas för anropsstack.

I tolken finns ett pass där vi går från ett syntaxträd där alla variabler
antas substitueras (eller ligga på heapen) till ett träd där lokala variabler
istället har information om sin position i aktiveringsposten, vilket nu ska presenteras.


%%%%%%%%%%%%%%%%%%%%%%%%%%%%%%%%%%%%%%%%%%%%%%%%%%%%%%%%%%5


\subsection{Lokaliseringstransformation}

För att anropsstacken ska fungera behöver varje fri variabel 
markeras med vilken position den kommer få
i aktiveringsposten. Det betyder att den behöver ett heltalsindex som anger
vilket offset den har. Det behövs en transformation som gör detta före koden 
körs. Nu ska lokaliseringstransformationen presenteras på funktionen \ic{map}, som
definieras på följande sätt:

% Genom exempel ska vi förklara hur lokaliseringstransformationen går till.

\begin{codeEx}
map f xs = case xs of
    { Cons y ys -> let
        { a = THUNK (f y)
        ; b = THUNK (map f ys)
        ; c = CON (Cons a b)
        } in c
    ; Nil -> xs};
\end{codeEx}

Eftersom argumenten skall ligga i aktiveringsposten ger vi dem indicier som anger deras 
offset. Detta skrivs med \ic{<x,n>} där \ic{x} är vad variabeln kallades 
förut och \ic{n} dess offset.

\begin{codeExDiff}
map |<f,0>| |<xs,1>| = case |<xs,1>| of
    { Cons y ys -> let
        { a = THUNK (|<f,0>| y)
        ; b = THUNK (map |<f,0>| ys)
        ; c = CON (Cons a b)
        } in c
    ; Nil -> |<xs,1>|};
\end{codeExDiff}


Eftersom variabler som binds med både \ic{case} och \ic{let} 
ska använda aktiveringsposten så ska även dessa ges indicier.

\begin{codeExDiff}
map <f,0> <xs,1> = case <xs,1> of
    { Cons |<y,2>| |<ys,3>| -> let
        { |<a,4>| = THUNK (<f,0> |<y,2>|)
        ; |<b,5>| = THUNK (map <f,0> |<ys,3>|)
        ; |<c,6>| = CON (Cons |<a,4>| |<b,5>|)
        } in |<c,6>|
    ; Nil -> <xs,1>};
\end{codeExDiff}

Extra information krävs för  \obj{THUNK}-objekten som kommer att evalueras någon 
annanstans, och därmed i en annan aktiveringspost. Det betyder att indicierna som 
används utanför \obj{THUNK}-objekten i \ic{map} inte gäller i thunkarna. Vi kommer därför att evaluera
thunkar i en egen aktiveringspost. Det kan dock vara så att thunken ändå använder sig
av de lokala variabler som finns i den yttre funktionen. I exemplet har vi till exempel att 
\ic{<a,4> = THUNK (<f,0> <y,2>)} där f och y är kommer utifrån. Därför 
måste \obj{THUNK}-objekten ha information om vilka lokala variabler i \ic{map} 
som de använder, så att de kan läggas på dess aktiveringspost. 

Noterbart är även att när \kw{let} allokerar kommer den att byta ut alla lokala variabler
mot vad de refererar till i aktiveringsposten. Vi får följande: 

%(Cons pekar på ett heapobjekt och ligger alltså inte i aktiveringsposten):

\begin{codeExDiff}
map <f,0> <xs,1> = case <xs,1> of
    { Cons <y,2> <ys,3> -> let
        { <a,4> = THUNK |[<f,0>, <y,2>]|  (|<f,0> <y,1>|)
        ; <b,5> = THUNK |[<f,0>, <ys,3>]| (map |<f,0> <ys,1>|)
        ; <c,6> = CON (Cons <a,4> <b,5>)
        } in <c,6>
    ; Nil -> <xs,1>};
\end{codeExDiff}

Konstruktorn \ic{Cons} får inget index eftersom det inte är en variabel
utan ett namn för att kunna särskilja den från andra konstruktorer. 

I \ic{<a,4> = THUNK [<f,0>, <y,2>]  (<f,0> <y,1>)} ser vi att 
\ic{f} och \ic{y} har fått nya index jämfört med vad de
hade utanför thunken, vilket beror på att de som sagt kommer att evalueras
med en egen aktiveringspost.

\begin{comment}
Det största värdet på variabelindex plus ett blir funktionens
aktiveringspoststorlek, vilken man alltså kan veta statiskt
(vid kompileringstid i en kompilator). Storleken för \ic{map}-funktionen blir alltså 7.
\end{comment}

Alla funktioner har nu information om storleken på sin aktiveringspost, så att 
den kan allokeras när funktionen ska köras.
Detta möjliggör en effektiv implementering som använder arrayer av konstant 
storlek, vilket ger en konstant uppslagstid.
Även thunkar kan statiskt veta sin posts storlek,
som räknas ut på ungefär samma sätt som
för funktioner, där de utomstående lokala variablerna (de så kallade fria 
variablerna) kan ses som funktionsargument.
Thunken \ic{THUNK [<f,0>, <y,2>]  (<f,0> <y,1>)} behöver storlek 2.

Ett fall som är lurigt att implementera är \kw{letrec}.
Följande funktion \ic{f} ger tillbaka en bool som säger om 
argumentet \ic{n} är jämnt eller inte. Det är förvisso ett
ineffektivt sätt att beräkna detta på, men illustrativt för denna diskussion,
då de är ömsesidigt rekursiva.

Så här ser funktionen ut skriven i vårt sockerspråk, fast med thunkar som i STG:

\begin{codeEx}
f n = letrec 
    { even x = THUNK (if (x == 0) True  (odd  (x - 1)))
    ; odd  x = THUNK (if (x == 0) False (even (x - 1)))
    } in even n;
\end{codeEx}

Funktionen lambdalyfts (se avsnitt \ref{sec:LamLift}) först till följande:

\begin{codeEx}
f n = letrec 
    { even = THUNK (even' odd)
    ; odd  = THUNK (odd' even)
    } in even n;
    
even' odd x = THUNK (if (x == 0) True  (odd  (x - 1)));
odd' even x = THUNK (if (x == 0) False (even (x - 1)));
\end{codeEx}

Här måste \ic{even} ha \ic{odd}s index och \ic{odd} ha \ic{even}s index. Vi får följande:

\begin{codeEx}
f <n,0> = letrec 
    { <even,1> = THUNK [<even,1>,<odd,2>] (even' <odd,1>)
    ; <odd,2>  = THUNK [<even,1>,<odd,2>] (odd' <even,0>)
    } in <even,1> <n,0>;
\end{codeEx}

Den observanta läsaren ser att i första thunken läggs \ic{even} till utan att den används. 
Detta är ett implementationsval som tillåter STG att alltid lägga till allt som 
binds i en \ic{letrec} i början i aktiveringsposterna när thunkarna ska evalueras. 
Att mer precist hålla reda på vilka thunkar som behöver vad är svårare att implementera.


%%%%%%%%%%%%%%%%%%%%%%%%%%%%%%%%%%%%%%%%%%%%%%%%%%%%%%%%%%%%%%%%


\subsection{Evaluering med anropsstacken}



Nu ska vi visa hur anropsstacken kan se ut vid ett anrop till \ic{map}.
Funktionen \ic{enumFrom} skapar en oändlig lista, \ic{[x, x+1, x+2, ...]}.
 
\begin{codeEx}
enumFrom x = x : enumFrom (x + 1);

main = map cube (enumFrom 1);
\end{codeEx}

Main avsockras till:
\begin{codeEx}
main = let <list,0> = THUNK [] (enumFrom 1)
       in  map cube <list,0>
\end{codeEx}



Notera att \ic{list}-thunken inte har några fria variabler.
I evalueringen allokeras först denna thunk och sedan ska anropet till map
utföras. 
Argumenten pushas först på continuation-stacken så att den ser ut såhär:
\begin{codeEx}
Arg cube
Arg list
\end{codeEx}

Maskinen ser att \ic{map} tar två argument, och funktionsapplikationen 
kan köras utan att ett \obj{PAP}-objekt behöver skapas.

Vid funktionsapplikation tas den nuvarande aktiveringsposten bort och ersätts 
med funktionens aktiveringspost. Detta är för att alla anrop betraktas vara
svansanrop.
När det sista uttrycket som utförs i en funktion är ett funktionsanrop, är
detta ett svansanrop. De fall som inte är svansanrop
är de som är i casegranskaren. Dessa kommer att återvända till den anropande
funktionen för att sen välja rätt gren. Vi kommer senare att se att vi i detta
fall duplicerar den översta aktiveringsposten. 
Att \kw{let}-binda resultatet av ett
funktionsanrop är inte heller ett svansanrop, för att det är faktiskt inte ett
anrop alls! Som bekant senareläggs det anropet i en thunk. Detta kan sen vara
del av ett svansanrop, om denna thunk returneras från någon funktion, eller inte,
om denna thunks innehåll forceras av en \kw{case}-granskare.

Tillbaka till anropet till \ic{map}. 
Argumenten läggs i en ny aktiveringspost, ersätter den som \ic{main} har skapat,
och maskinen går in i funktionen. 
Aktiveringsposten blir:
\begin{codeEx}
[cube, list]
\end{codeEx}
I definitionen av \ic{map} såg vi förut att \ic{f} har index 0 och
\ic{xs} index 1, och de har nu fått värden.

Det ska utföras en \kw{case}-granskning på \ic{<xs,1>} och på position 1 i
aktiveringsposten ligger \ic{list}. Aktiveringsposten dupliceras eftersom
det är i en \kw{case}-granskare. Detta är för att när maskinen senare kommer
att återvända från att ha evaluerat granskaren kommer den att ta bort den 
översta posten. Maskinen börjar evaluera \ic{list}, och anropstacken blir:
\begin{codeEx}
[cube, list]
[cube, list]
\end{codeEx}

\ic{list} pekar på den ovan definierade thunken, och maskinen kommer evaluera den.
Detta räknas som ett svansanrop och den översta aktiveringsposten byts
ut mot thunkens tomma.

Anropstacken blir:
\begin{codeEx}
[]
[cube, list]
\end{codeEx}

På koden ligger nu \ic{enumFrom 1} från thunken. 
Detta kommer efter några steg att evalueras till en atom 
%, låt oss kalla den \ic{l}, 
som pekar på 
konstruktorn \ic{Cons 1 list'}, där list' är
en thunk som kommer att evaluera \ic{enumFrom 2}. 
% Denna atom\ic{l} 
Atomen
returneras. Vid returnering antas som bekant svansanrop och 
den översta aktiveringsposten poppas.


Nu är det känt i \ic{map}-funktionen vad \ic{<xs,1>} är för konstruktor, nämligen
\ic{Cons 1 list'}. Den grenen väljs och då pushas \ic{1} och 
\ic{list'} på aktiveringsposten. 

Anropstacken blir:
\begin{codeEx}
[cube, list, 1, list']
\end{codeEx}

När variabler bundna till \kw{let}-thunkar allokeras, som \ic{<a,4>}
i \ic{map}, skapas ett heapobjekt, men inte med samma aktiveringspost.
Istället görs ett uppslag av varje variabel i thunkens variabellista, och
sen allokeras thunken med en anropsstack efter dessa uppslag. Hur går 
\kw{let}-bindningen till i \ic{map}? Så här ser koden ut:

\begin{codeEx}
map <f,0> <xs,1> = case <xs,1> of
    { Cons <y,2> <ys,3> -> let
        { <a,4> = THUNK [<f,0>, <y,2>]  (<f,0> <y,1>)
        ; <b,5> = THUNK [<f,0>, <ys,3>] (map <f,0> <ys,1>)
        ; <c,6> = CON (Cons <a,4> <b,5>)
        } in <c,6>
    ; Nil -> <xs,1>};
\end{codeEx}

Så \ic{<a,4>}-thunken  vill ha index $0$ och $2$ från
aktiveringsposten. Det är \ic{cube} och \ic{1}, precis anropen till
\ic{cube 1}. 
Thunken som allokeras kommer alltså se ut så här
\ic{THUNK [cube,1] (<f,0> <y,1>)}.
Om den senare kommer att evalueras skapas aktiveringsposten \ic{[cube,1]}.


Sammanfattningsvis ersätts den översta aktiveringsposten vid 
evaluering av funktioner och thunkar. När en \kw{case}-granskare börjar evalueras
dupliceras den översta aktiveringsposten. Att returnera från en funktion,
och för STG-maskinen innebär det att returnera från en casegranskare,
resulterar i att poppa den översta aktiveringsposten. Då pushas också 
de variabler som rätt gren i \kw{case}-satsen binder på posten.  




%Gå in i casegranskare - Duplicering

%Väljer branch - lägger till grejs på aktiveringsposten
   
%Returnera från en funktion - Ta bort översta stackframe

%Gå in i en funktion/thunk - Skapa en ny stackframe // Tailcall


 


\end{document}
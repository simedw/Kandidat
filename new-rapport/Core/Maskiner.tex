\documentclass[../Core]{subfiles}
\begin{document}


% Var ska det här in istället?
% Att skriva en kompilator hade varit ett delikat arbete i sig, även utan
% optimering. Vi ville fokusera på optimeringen och valde därför att skriva en tolk.


\subsection{Olika maskiner}
\label{sec:Core:Mask}

För att evaluera program skrivna i ett funktionellt programmeringsspråk
behövs en evalueringsmodell. Det finns olika abstrakta maskiner utvecklade för
detta ändamål, till exempel G-maskinen, TIM (Three Instruction Machine), GRIN \cite{grin} och STG.
Dessa går att dels tolka, det vill säga att i något programmeringsspråk (exempelvis
Haskell) skriva ett program som kopierar maskinens funktionalitet. Det går också att
kompilera dem till något lågnivåspråk som C, assemblyspråk eller LLVM.


Av de olika maskiner vi undersökte ansåg vi att STG var lämpligast. I G-maskinen och TIM så är
instruktionerna på låg nivå, liknande assemblerinstruktioner men för att manipulera
ett tänkt syntaxträd. STG och Grin, är å andra sidan, mer som ett funktionellt
programmeringsspråk vilket skulle underlätta arbetet på optimeringen.

STG är förhållandevis enkel att implementera, och är tillräckligt explicit för att
tillåta oss göra de utökningar för att kunna optimera under körningstid som vi vill
göra. Medan många av Grin-maskinens fördelar kommer från att den känner till alla funktioner i hela
programmet (funktioner i andra moduler kopieras in när de används). Den här informationen
använder Grin sedan för att göra en stor graf över vilka funktioner som kallar på
vilka andra funktioner. Detta gör att vi tror att det blir enklare att jobba med STG 
som är mer traditionell.
% skriva något om STG
% omstrukturera
% koens kommentar:
%   står mycket om grin, lite om STG, flytta?

\end{document}

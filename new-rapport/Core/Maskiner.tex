\documentclass[../Core]{subfiles}
\begin{document}


% Var ska det här in istället?
% Att skriva en kompilator hade varit ett delikat arbete i sig, även utan
% optimering. Vi ville fokusera på optimeringen och valde därför att skriva en tolk.


\section{Olika maskiner}
\label{sec:Core:Mask}

För att evaluera program skrivna i ett funktionellt programmeringsspråk
behövs en evalueringsmodell. Det finns olika abstrakta maskiner utvecklade för
detta ändamål, till exempel G-maskinen \cite{G-machine}, TIM\footnote{Three Instruction Machine}\cite{TIM}
, Grin \cite{grin} och STG.
Dessa går att dels tolka, det vill säga att i något programmeringsspråk (exempelvis
Haskell) skriva ett program som simulerar maskinens funktionalitet. Det går också att
kompilera dem till ett lägre språk som C, assemblyspråk eller LLVM\footnote{http://llvm.org/}.


Av de olika maskiner som övervägdes ansågs STG var lämpligast. I G-maskinen och TIM  är
instruktionerna på låg nivå, liknande assemblerinstruktioner, som manipulerar
ett tänkt syntaxträd. STG och Grin, är å andra sidan mer som funktionella
programmeringsspråk vilket skulle underlätta arbetet på optimeringen.

STG-maskinen är förhållandevis enkel att implementera, och är tillräckligt flexibel för att
tillåta de utökningar som optimering under körningstid kräver.
%för att kunna optimera under körningstid som vi vill göra.
Många av Grin-maskinens optimeringar kommer från att den känner till alla funktioner i hela
programmet, och funktioner i andra moduler kopieras in när de används. Den här informationen
använder Grin sedan för att göra en stor graf över vilka funktioner som kallar på
vilka andra funktioner. Eftersom Grin arbetar på det sättet tror vi att det blir enklare att jobba med STG 
som är mer traditionell.
% skriva något om STG
% omstrukturera
% koens kommentar:
%   står mycket om grin, lite om STG, flytta?

\end{document}

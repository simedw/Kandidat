\documentclass[../Core]{subfiles}
\begin{document}

\subsection{Olika maskiner}
För att evaluera program skrivna i ett lat funktionellt programmeringsspråk
behövs en evalueringsmodell. Det finns olika abstrakta maskiner utvecklade för 
detta ändamål, tex G-maskinen, TIM (Three Instruction Machine), GRIN, och STG.
Dessa går att dels tolka, dvs skriva ett program som kopierar maskinens
funktionalitet i något programmeringsspråk, exempelvis Haskell. Man kan också
kompilera dem till något lågniva-språk som C, assembly eller LLVM.
    I det här arbetet har vi valt att utveckla en egen tolk. Att skriva en
kompilator hade varit ett delikat arbete i sig, även utan optimering. Vi ville
fokusera mer på optimeringen och valde därför att skriva en tolk.
    Av de olika maskiner vi undersökte när vi valde vilken vi skulle tolka så
ansåg vi att STG var mest lämpad. I de andra ovan nämnda maskinerna är
instruktionerna lågnivå, liknande assemblerinstruktioner men för att manipulera
ett tänkt syntaxträd. STG, å andra sidan, ser ut som ett funktionellt
programmeringsspråk och det skulle underlätta att arbeta på optimeringen av
den anledningen.


\end{document}

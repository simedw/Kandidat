\documentclass[../Core]{subfiles}
\begin{document}

\subsection{Olika maskiner}
För att evaluera program skrivna i ett funktionellt programmeringsspråk
behövs en evalueringsmodell. Det finns olika abstrakta maskiner utvecklade för
detta ändamål, till exempel G-maskinen, TIM (Three Instruction Machine), GRIN och STG.
Dessa går att dels tolka, det vill säga att i något programmeringsspråk, exempelvis
Haskell, skriva ett program som kopierar maskinens funktionalitet. Det går också att
kompilera dem till något lågniva-språk som C, assembly eller LLVM.

Att skriva en kompilator hade varit ett delikat arbete i sig, även utan
optimering. Vi ville fokusera mer på optimeringen och valde därför att skriva en tolk.

Av de olika maskiner vi undersökte när vi valde vilken vi skulle tolka
ansåg vi att STG var lämpligast. I de andra ovan nämnda maskinerna är
instruktionerna på låg nivå, liknande assemblerinstruktioner men för att manipulera
ett tänkt syntaxträd. STG, å andra sidan, ser ut som ett funktionellt
programmeringsspråk vilket skulle underlätta arbetet på optimeringen.


\end{document}

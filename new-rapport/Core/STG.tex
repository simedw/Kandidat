\documentclass[../Core]{subfiles}

\begin{document}
\subsection{STG-språket}

% Blah den här filen används ju inte längre

Så här ser BNF-grammatiken ut för corespråket:


\begin{equation*}
\begin{aligned}
\text{Variables} \quad & x,f \\
\text{Constructors} \quad & C \\
\defBNF{Literals}{lit}{i \; | \;  d \;|\; c}{Primitiv datatyp} \\
\defBNF{Atoms}{a, v}{lit \; | \; x}{Atomer för funktionsanrop etc} \\
\defBNF{Expressions}{e}{a}{Atom} \\
    \defaltBNF{f \; \overline{a}}{Funktionsanrop} \\
    \defaltBNF{\oplus \; \overline{a}}{Primitiv operation} \\
    \defaltBNF{\mathtt{let \{} \overline{x = obj} \mathtt{ \} in } \; e}{Lokal allokering} \\
    \defaltBNF{\mathtt{letrec \{} \overline{x = obj} \mathtt{ \} in } \; e}{Lokal rekursiv allokering} \\
    \defaltBNF{\mathtt{case} \; e \;  \mathtt{ of \{ } \overline{alt} \mathtt{\}}}{Evaluering av uttryck} \\
\defBNF{Alternative}{alt}{C \; \overline{x} \rightarrow e}{Matcha mot konstruktor} \\
    \defaltBNF{x \rightarrow e}{Matcha med en variabel} \\
\defBNF{Heap Object}{obj}{FUN(\overline{x} \rightarrow e)}{Funktionsobjekt} \\
    \defaltBNF{PAP(f \; \overline{a})}{Partielltapplicerad funktion} \\
    \defaltBNF{CON(C \; \overline{a})}{Konstruktorer} \\
    \defaltBNF{THUNK \; e}{Senarelagt uttryck} \\
    \defaltBNF{BLACKHOLE}{Används för evaluering} \\
\defBNF{Programs}{prog}{\overline{f=obj}}{}
\end{aligned}
\end{equation*}

STG-språket är som en mer primitiv form av vårt sockerspråk, där fler saker är mer explicita
än vad de är i sockret. Detta gör det enklare
att förstå vad språkkonstruktionerna egentligen gör på en lägre nivå och hur de skall tolkas. Här kommer in kort introduktion
och senare 

\end{document}

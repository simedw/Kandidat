\documentclass[Rapport]{subfiles}
\begin{document}

\section{Corespråk}

\NOTE{ Ett första försök till en övergångstext. }

I förra avsnittet introducerades sockerspråket, ett litet men användbart
funktionellt programmeringsspråk, det som används genomgående i det här
arbetet för att skriva lättläst kod.
    I detta avsitt ska vi se hur denna kod tolkas. För att underlätta
tolkningen översätts koden till ett annat språk, corespråket. I avsitt
% här borde man kunna ha någon magisk latex-referens antar jag
3.2, \emph{En semantik för STG} förklaras reglerna som används för
tolkningen, i nästa avsnitt beskrivs hur sockerspråket översätts
till STG, och också hur lambdafunktioner lyfts upp till toppninvå.
I avsnitt 3.4 kan du läsa om hur primiva datatyper boxas, och hur IO
går till i tolken. Nästföljande avsnitt handlar om anropsstacken, som
används för att spara undan lokala variabler i en funktion.

\subfile{Core/Maskiner}

% \subfile{Core/STG}

\subfile{Core/Semantik}

\subfile{Core/Avsockring}

\subfile{Core/Extensions}

\subfile{Core/Callstack}

\end{document}

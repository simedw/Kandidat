\documentclass[Rapport]{subfiles}
\begin{document}

\section{Corespråk}
\label{sec:Core}

%\NOTE{ Ett första försök till en övergångstext. }

\overviewCore


I förra avsnittet introducerades sockerspråket, ett litet men användbart
funktionellt programmeringsspråk, det som används genomgående i det här
arbetet för att skriva lättläst kod.
    I detta avsnitt ska vi se hur koden tolkas. För att underlätta
tolkningen översätts koden till ett annat språk, corespråket. 
% här borde man kunna ha någon magisk latex-referens antar jag
Huvudavsnittet för den här texten är \ref{sec:SemStg}, \emph{En semantik för STG}
som förklaras reglerna som används för
tolkningen. 

Därpå följande avsnitt beskrivs olika verktyg som används för att
underlätta arbetet med det här språket och hur tolken är uppbyggd,
som avsnitt \ref{sec:Diabetes} om avsockringen från sockersrpåket och 
sektion \ref{sec:LamLift} 
om hur lambdafunktioner lyfts upp till toppnivå, därefter
boxning av primitiva datatyper, samt vilken modell av IO tolken använder.

I avsnitt \ref{sec:CallStack}, beskrivs \emph{anropsstacken}, som förklarar hur tolken
sparar undan lokala variabler i en funktion. Anropsstacken är en 
stor prestandahöjare över substitution, och därför är det här också ett
viktigt avsnitt i den här sektionen.

\subfile{Core/Maskiner}

\subfile{Core/Semantik}

\subfile{Core/Avsockring}

\subfile{Core/Extensions}

\subfile{Core/Callstack}

\end{document}

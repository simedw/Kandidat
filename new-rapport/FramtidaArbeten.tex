\documentclass[Rapport]{subfiles}
\begin{document}

\chapter{Slutsats och framtida arbete}

Vi har i denna rapport presenterat en utökning av STG-språket med ett nytt objekt, \obj{OPT}, för att annotera kod som ska optimeras under körningstid. Vi har även ett här presenterat tillhörande evalueringschema som bevarar STG:s lata semantik. 

En avgränsning för vårt projekt var att jobba inom ramen för en tolk. 
Detta av flera anledningar, men främst för att det är lättare att göra prototyper för 
olika regler för optimise. Nu när vi har tagit fram fungerande optimeringssemantik 
är nästa steg att implementera vårt språk i en kompilator.
Med anropsstack som vi bytte till istället för substitutionsfunktionen tror vi att implementationen
i en kompilator skulle vara realiserbar.


Vi använde STG, som är ett av de interna språken som GHC arbetar med i sina
mellansteg Kompilatorn YHC\footnote{http://www.haskell.org/haskellwiki/Yhc} 
har också ett liknande mellanspråk.
Det hade varit intressant att utöka vårt språk så att det skulle kunna 
användas som backend till någon större kompilator, eller något runtime-system.
Vi har hela tiden haft detta i åtanke och det är delvis därför vi ej har någon 
typcheckare, eftersom det redan förväntas vara gjort vid skedet då vårt arbete kommer in.
En typcheckare och stöd för bland annat algebraiska datatyper och typklasser i en 
frontend hade dock varit trevligt då det är lätt att skriva fel när man skriver
program till vårt språk. 


Vi har inte formellt bevisat att vår utökning optimerar korrekt. Dock klarar vår 
referensimplementation, tolken, en mängd olika testfall vilket ger oss stort förtroende för 
dess korrekthet.
Ett steg för att förvissa sig om korrektheten skulle vara att göra 
QuickCheck\footnote{http://www.cs.chalmers.se/\textasciitilde rjmh/QuickCheck/}-egenskaper 
och generatorer för att se att en optimerad funktion gör samma sak som en ooptimerad. 
Ett annat sätt att säkerställa att den
gör vad som förväntas skulle vara att formalisera reglerna och bevisa
att de är korrekta, exempelvis med hjälp av en bevisassistent som 
Agda\footnote{http://wiki.portal.chalmers.se/agda/}.

\section{Utveckling av optimeringen}

Inledande prestandatester visar i ett par väl valda exempel upp emot åtta gångers
tidsvinst. Dock finns det fall där vi inte får någon väsentlig förbättring eller
till och med en försämring. Vi har ännu inte testat vår implementation på något riktigt stort
projekt. 

Optimeringsfunktionaliteten som presenteras i den här rapporten är inte
komplett. Det finns mycket utrymme för att lägga till fler optimeringar
som skulle kunna förbättra koden ytterligare. Det som ligger nära till hands är
delning, smartare hantering av rekursiva funktioner och tvingad strikthet.

Dessa är genomförbara utvecklingar, och ska inte ses som omöjliga önskningar.
Anledningen till att de inte finns med och är implementerade är tidsbrist, vilket
framtvingade avgränsningar.

\paragraph{Delning}
Den powerfunktion som vi har fokuserat på i det här arbetet har linjär
komplexitet, men det finns som bekant också en logaritmisk, nämligen följande:

\begin{codeEx}
power n x = case n == 0 of
    { True  -> 1
    ; False -> case even n of
        { True  -> let a = power (n / 2) x 
                   in  a * a
        ; False -> x * power (n - 1) x
        }
    };
\end{codeEx}

Optimeringssemantiken, som den ser ut nu, kommer att infoga \ic{a} två gånger,
liknande call-by-name, men inte call-by-need. Anledningen är att beräkningen
\ic{a} lagras i en thunk på abyssen, och när senare multiplikationen ska
infogas infogas även \ic{a}, och detta två gånger. Det resultatet sparas
inte.

    Just nu är det svårt att återskapa thunkar efter att de optimerats både på grund
av att de kan refereras till i andra thunkar och konstruktorer, och på grund
av anropsstacken. Detta problem är inte löst.

\paragraph{Rekursiva funktioner}
\label{sec:future-regexp}
Ibland är inte utrullning det bästa, som i fallet med reguljära uttryck.
Betrakta funktionen \ic{match :: RegExp -> String -> Bool}, som givet
ett reguljärt uttryck och en sträng ger tillbaka en boolean som talar om huruvida strängen
satisfierar uttrycket. Om man skulle rulla ut funktionen på den okända strängen
skulle man få en godtyckligt stor funktion, eftersom strängen kan vara 
obegränsat lång.

    Istället vill man specialisera \ic{match}-funktionen på olika reguljära uttryck,
som i sin tur kan anropa varandra. Som det är nu kan inte funktionen som
optimeras anropa sig själv, eller andra funktioner som den har skapat.

    I det här fallet vill man skapa specialiserade match-
funktioner för olika reguljäruttryck, 
som \ic{(a.b)*}, \ic{a.b}, \ic{a} och \ic{b}, och att de i sin tur
kan anropa varandra.

\paragraph{Tvingad strikthet}
I vissa fall vill man tvinga fram strikthet, som i fallet 
\ic{optimise (sum . zipWith (*) getIntList)} som skulle kunna bli en utrullad,
optimerad skalärprodukt. Just nu sker inte detta eftersom anropet till nästa
\ic{zipWith} är lat då det skapar ett \ic{Cons}-objekt, där elementet är de första
elementen i listorna adderat, och resten av listan är ihopzippningen av resten
av listorna. \ic{Cons} ska inte evaluera sina argument, och därför stannar
optimeringen. Om rekursiva funktioner skulle vara implementerade som nämnt ovan
skulle den kunna göra ett anrop till sig själv igen.

    I fallet med skalärprodukt vill man hellre att hela \ic{getIntList}-listan 
utrullas. Då behöver man någon sorts strikthetsnotation, antingen i koden 
för \ic{zipWith}, eller som argument till \kw{optimise}, t.ex. att \ic{Cons} ska vara strikt i 
sitt andra argument.

\paragraph{Välkänd case-lag}
Som visas i resultatet, och diskuteras i kapitel
\ref{sec:diskussion_om_exponentialitet} ger den nuvarande implementationen
av välkänd caselag tillsammans med \kw{casebranch}-optimeringen 
en komplexitet på optimeringen som inte är linjär. Reglerna för när denna
omskyffling ska ske kan justeras till linjär komplexitet. Exempelvis skulle man
kunna flagga att en gren inte behöver optimeras igen efter omskyfflingen. 
Något sådant sätt att markera grenar finns inte just nu.



\end{document}

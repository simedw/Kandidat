\documentclass[Rapport]{subfiles}
\begin{abstract}

Vi har utvecklat en lat semantik för optimering av partiellt applicerade funktioner 
under körningstid i ett funktionellt språk. För att möjliggöra detta har vi även implementerat en tolk 
med STG som corespråk. STG används även i Haskellkompilatorn GHC som en intern 
representation av koden i kompileringsfasen.
% använder sig av STG, som används i exempelvis Haskellkompilatorn GHC, som
%intern representation av koden. 
I vårt språk ges möjligheten att annotera var koden
ska optimeras under körningstid, vilket möjliggör fler optimeringar än de som kan
göras statiskt.
Vi introducerar tolken, de byggstenar den består av och STG grundligt och ger 
sedan en formell modell över hur optimeringen kan göras under körningstid.
Vi visar sedan våra praktiska resultat och ser att optimeringen
ger stora tidsvinster i de fall då en funktion som partiellt applicerats med värden, som inte är kända
statiskt, optimeras och sedan körs ett relativt stort antal gånger.

\NOTE{
Detta är en $\beta$-rapport.
}
\end{abstract}
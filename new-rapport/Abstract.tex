\documentclass[Rapport]{subfiles}
\begin{abstract}

%Vi har utvecklat en lat semantik för optimering av partiellt applicerade funktioner 
%under körningstid i ett funktionellt språk. För att möjliggöra detta har vi även 
%implementerat en tolk med STG som corespråk.

%, som använts i Haskellimplementationer.

%STG används även i Haskellkompilatorn GHC som en intern 
%representation av koden i kompileringsfasen.

% använder sig av STG, som används i exempelvis Haskellkompilatorn GHC, som
%intern representation av koden. 




% Vi har gjort en utokning av STG for optimering av partiellt applicerade funktioner under korningstid i ett lat funktionellt sprak. Vi konkretiserat detta med att utveckla en tolk.

Vi har utvecklat en tolk med möjligheten att optimera partiellt applicerade
funktioner under körningstid i ett lat funktionellt språk. 

Som corespråk använder tolken STG, som också används i Haskellimplementationer.

Språket har modifierats för att användaren ska kunna annotera var koden
ska optimeras under körningstid. Detta visar sig ge fler möjligheter till
optimeringar än de som kan göras statiskt.

Vi introducerar tolken, de byggstenar den består av och STG grundligt och ger 
sedan ett formellt schema som beskriver semantiken för hur optimeringen görs under 
körningstid. Sedan visas de praktiska resultat som fås: optimeringen ger i vissa fall stora 
tidsvinster då en funktion som partiellt applicerats med värden, 
som inte är kända statiskt, optimeras och körs ett relativt stort antal gånger.

\end{abstract}
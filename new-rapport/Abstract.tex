\documentclass[Rapport]{subfiles}
\begin{abstract}

Vi har utvecklat en tolk för ett funktionellt programmeringsspråk med lat semantik.
Tolken använder sig av STG, som används i exempelvis Haskellkompilatorn GHC, som
intern representation av koden. I språket ges möjligheten att annotera var koden
ska optimeras under körningstid, vilket möjliggör fler optimeringar än de som kan
göras statiskt.

Vi introducerar tolken, de byggstenar den består av och STG grundligt och ger 
sedan en formell modell över hur optimeringen kan göras under körningstid.

Vi visar sedan våra praktiska resultat och ser att optimeringen
ger stora tidsvinster i de fall då en funktion som partiellt applicerats med värden, som inte är kända
statiskt, optimeras och sedan körs ett relativt stort antal gånger.

\NOTE{
Detta är en $\alpha$-rapport. Till nästa version, $\beta$, ska vi ha:

\begin{itemize}

\item Casestudy, där vi visar hur man använder optimise i ett mer riktigt exempel
\item Resultat (mer information i noteringen där)
\item Förklara hur optimise fungerar med anropsstacken
\item Fixa till relaterade och framtida arbeten
\item Appendix (mer information i noteringen där)
\item Korrekturläst allt ordentligt (speciellt andra halvan av rapporten behöver detta)

\end{itemize}
}
\end{abstract}
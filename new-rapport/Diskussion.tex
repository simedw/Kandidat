\documentclass[Rapport]{subfiles}
\begin{document}

\section{Diskussion}
Vi har gjort en tolk för ett funktionellt programmeringsspråk
med en semantik som följer STG-maskinen som används i Haskellkompilatorn GHC.
Till denna har vi lagt till möjligheten att optimera partiellt applicerade funktioner under körningstid
då vi känner till värdena som applicerades och kan utnyttja dessa till optimering.

\subsection{Implementerbarhet i kompilator}
Vi valde att göra en tolk eftersom det ofta är enklare än att implementera en kompilator.
En tolk innebär att man känner till syntaxträdet hos koden, vilket vanligtvis kompilerad
kod inte gör, vilket gör att om det skulle implementeras i en kompilator skulle
denna information behöva kompileras ner för att finnas tillgänglig när programmet körs.

\subsection{Lathet}
Vid optimeringen vill man att koden ska vara så optimerad som möjligt, men man får
vara försiktig så att man inte är för aggressiv, eftersom det kan leda till icke-terminering.

\begin{codeEx}
take n xs = let
    { a1 = THUNK (n == 0)
    ; a2 = THUNK (case xs of
        { Nil -> Nil
        ; Cons _ ys -> let
            { b1 = THUNK (n - 1)
            } in take b1 ys
        )
    } in if a1 Nil a2;

main = (optimise (take 5)) [1,2,3,4,5,6,7];
\end{codeEx}

I exemplet ovan går optimisefunktionen in och optimerar a2 oberoende av värdet på
a1, och eftersom vi inte vet värdet på xs när funktionen optimeras leder det till
icke-terminering om man inlinar det rekursiva anropet till take.

\subsection{Optimise-with}
En möjlig lösning på problemet ovan är att låta programmeraren själv avgöra hur
många gånger en funktion får inlinas.

\begin{codeEx}
main = optimise (take 5) with { inline take 5 } [1,2,3,4,5,6,7];
\end{codeEx}

I exemplet ovan betyder inline take 5 att take bara får inlinas 5 gånger.
Denna lösning kan tyckas vara suboptimal, och man skulle hellre vilja att optimise
kunde vara lite smartare genom att detektera när den potentiellt skulle kunna
falla in i en oändlig inliningloop. Men det har vi ännu inte funnit
någon lösning på.

\subsection{Sharing}
Ett annat problem med lathet är att optimise måste se till att variabler bara 
evalueras en gång. Detta ställer till besvär hur man skall optimera let satser.

\begin{codeEx}
test x y = let
    { a = THUNK (fib x + y)
    } in (+) a a;

main = optimise (test 100) 5
\end{codeEx}

I exemplet ovan inlinas a i let-satsen, vilket gör att optimise sedan kommer att
köras två gånger på samma kod, när man egentligen skulle vilja att den delar resultatet
av att optimera a och använder det på båda ställen där det används.

\begin{codeEx}
test_x y = let
    { a = THUNK ([|fib x|] + y)
    } in (+) a a;
\end{codeEx}
Där [|.|] betyder att man evaluerat uttrycket till den kända x. 

\subsection{Call stack}

Från början använde vi substituering för att göra exempelvis funktionsapplikation
vilket betyder att man går ner i syntaxträdet och byter argumentvariablerna i funktionen
mot vad argumenten faktiskt är. Detta är långsamt och inte särskilt realistiskt - 
speciellt inte om man talar om en kompilator. Kompilatorer kan istället använda call-stacks
där man låter funktionsargumenten indexeras in på en stack, så att man slipper
gå ner i hela syntaxträdet (vilket ju vanligtvis inte är möjligt i kompilerad kod 
om den inte har tillgång till det).

För att göra vår tolk mer realistisk bytte vi till att använda detta, vilket också
gav oss prestanda som är mer jämförbar med det riktiga livet.

\begin{codeEx}
f a = case g a b of
    A -> .. a ..
    B -> .. a ..
\end{codeEx}

Om vi vill evaluera vad f 5 blir så substituerar vi argumentet a mot 5 i kroppen.

\begin{codeEx}
f_a = case g 5 b of
    A -> .. 5 ..
    B -> .. 5 ..
\end{codeEx}

Med en callstack så har man gjort att fuktionen f har sina lokala variabler indexerade
så istället för a har man 0 för första lokala variablen (i exemplet visas det som
<a,0> för att underlätta läsning). Detta görs för alla lokala variabler, och inte
bara argument, även let-bundna samt varibler bundna via case blir indexerade.

Att ge alla lokala variabler ett index görs i ett transformeringspass före tolken körs,
så att inget onödigt arbete behöver göras vid körningstid.

\begin{codeEx}
f <a,0> = case (+) <a,0> b of
    A -> .. <a,0> ..
    B -> .. <a,0> ..
\end{codeEx}

När man anropar funktionen behöver man nu bara lägga argumentet a på position 0 på
stacken, och överallt där a används kan man sedan indexera in på stacken för att
få ut värdet.

\subsection{Optimise med call stack}

När vi implementerade call-stacks hade vi från början kvar vår gamla optimisefunktion
som fortfarande använde vanlig substituering. När optimise skulle köras fick vi då
transformera tillbaka syntaxträdet till den representationen, och sedan transformera
tillbaka när optimeringen var klar. Ganska ineffektivt.

[berätta om hur vi gjorde sedan]
\end{document}

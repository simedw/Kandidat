\documentclass{beamer}
\mode<presentation>{
  \usetheme{boxes}
}


\usepackage[T1]{fontenc}
\usepackage[utf8]{inputenc}
\usepackage[swedish]{babel}
\usepackage{epstopdf}

\usepackage{listings}
\usepackage{tikz}
\usetikzlibrary{trees,arrows,automata}

\usepackage{wasysym}
\usepackage{fancyvrb}
\DefineVerbatimEnvironment{code}{Verbatim}{fontsize=\small}
\DefineVerbatimEnvironment{example}{Verbatim}{fontsize=\small}
\newcommand{\ignore}[1]{
}

\lstnewenvironment{codeEx}
{\lstset{ basicstyle= \ttfamily }}
{}




\begin{document}

\title{Optimering under körningstid}


\author{Simon Edwardsson\smiley \and Olle Fredriksson\smiley
\and \linebreak{}
Daniel Gustafsson\smiley \and Dan Rosén\frownie}


\institute{\smiley Chalmers Tekniska Högskola\and \frownie Göteborgs Universitet}

\everymath{\displaystyle}

\begin{frame}
    \titlepage
\end{frame}

\begin{frame}

	Recap?

\end{frame}

\begin{frame}

% knyta an till f;rra presentationen utan att utesluta de nya, kort namna att vi optimerar saker och forklarar ett vi nu ska ge ett exempel dar det ar anvandbart

	En graf-ritare\\
	\pause
	$sin(x) + x$
	
	Om funktionen kommer fran anvandare?\\
	\pause
	Tree

	\pause
	Eval-funktionen.
	Stega igenom nagot...
	
	\pause
	Problembeskrivning + lsg

	\pause
	Till slut ett demo?

\end{frame}

\begin{frame}

	Vad sysslar vi egentligen med (vart befinner vi oss) (graf fran olle)

\end{frame}

\begin{frame}[fragile]
	Vad har hant sen sist?
	\pause
	Callstack
		Cool bild, forklarning
	\pause 
	Lathet, ge if exempel.

\begin{codeEx}
fact x = if (x == 0) 
    then 1
    else x * fact (x-1)
\end{codeEx}

\end{frame}

\begin{frame}
	Användningsområden.
\begin{itemize}
	\item Regular expressions
	\item Vetenskapliga beräkningar % dotproduct, power, newtonraphson, simuleringar
	\item Raytracer
\end{itemize}

Bild pa t ex newtonraphson kurva, nollpunkter och sant.

\end{frame}

\begin{frame}
	Focusera pa raytracer med resultat och bild pa alien.
	Fin bild.

\begin{figure}[H]
\centering
    \includegraphics[width=0.3\textwidth]{img/shapes.png} 
\end{figure}


\end{frame}
\begin{frame}
\begin{figure}[H]
\includegraphics[width=1\textwidth]{shapesnorm.eps}
\end{figure}


\end{frame}

\begin{frame}

	Lardomar, reflextioner

\end{frame}

\begin{frame}
	Begransningar 
	Sharing. Rec	
\end{frame} 

\begin{frame}
	Framtida arbeten?
	Mycket pepp!
	Riktig kompilator
	Olles bild del 3.
	
	$\ddot\smile$ 
\end{frame}


\newcommand{\ndist}{1.4cm}
\newcommand{\nshift}{0.8cm}
\newcommand{\ncol}{blue!20!white}

\begin{frame}
\begin{figure}[H]
    \centering
        \begin{tikzpicture}[->,node distance=\ndist, semithick]
            \tikzstyle{every state}=[fill=red!20!white,text=black]
            \node[state,fill=\ncol](add) {$+$};
            \node[state,fill=\ncol](mul)[below of=add, xshift=-\nshift] {$*$};
            \node[state,fill=\ncol](sin)[below of=mul, xshift=-\nshift] {sin};
            \node[state](x1)  [below of=sin, xshift=-\nshift] {$x$};
            \node[state](x2)  [below of=mul, xshift=\nshift]  {$x$};
            \node[state](five)[below of=add, xshift=\nshift]  {$5$};
            \path (add) edge node {} (mul)
                        edge node {} (five)
                  (mul) edge node {} (sin)
                        edge node {} (x2)
                  (sin) edge node {} (x1);
        \end{tikzpicture}
        \caption{ sin x * x + 5 }
    \end{figure}

\end{frame}

\end{document}
